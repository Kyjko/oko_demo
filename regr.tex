\documentclass[14p]{report}
\usepackage{amsfonts}
\usepackage{amsmath}
\usepackage{yfonts}
\usepackage{mathrsfs}
\usepackage{amssymb}
\usepackage[T1]{fontenc}
\usepackage{pgfplots}
\usepackage[margin=1in]{geometry}
\usepackage{xcolor}
\usepackage{pgfplotstable}
\usepackage{lmodern}
\usepackage{tikz}
\usepackage{titling}

\setlength{\droptitle}{-40em}
\pagestyle{headings}

\usetikzlibrary{calc}

\pgfplotsset{compat=newest}

\renewcommand*{\contentsname}{Tartalom}

\title{Lineáris regresszió elméleti összefoglaló}
\author{Bognár Miklós}
\date{Bevezetés az Ökonometriába}

\pagecolor{white}
\color{black}

\def\pmb{\boldsymbol}
\def\ebeta{\hat{\pmb{\beta}}}
\def\e{\epsilon}
\def\bar{\overline}

\selectcolormodel{gray}

\setcounter{chapter}{0}
\renewcommand{\chaptername}{}
\renewcommand{\thechapter}{}

% coverlogo

\newcounter{x}
\newcounter{y}
\newcounter{z}

% The angles of x,y,z-axes
\newcommand\xaxis{210}
\newcommand\yaxis{-30}
\newcommand\zaxis{90}

% The top side of a cube
\newcommand\topside[3]{
	\fill[fill=yellow, draw=black,shift={(\xaxis:#1)},shift={(\yaxis:#2)},
	shift={(\zaxis:#3)}] (0,0) -- (30:1) -- (0,1) --(150:1)--(0,0);
}

% The left side of a cube
\newcommand\leftside[3]{
	\fill[fill=cyan, draw=black,shift={(\xaxis:#1)},shift={(\yaxis:#2)},
	shift={(\zaxis:#3)}] (0,0) -- (0,-1) -- (210:1) --(150:1)--(0,0);
}

% The right side of a cube
\newcommand\rightside[3]{
	\fill[fill=gray, draw=black,shift={(\xaxis:#1)},shift={(\yaxis:#2)},
	shift={(\zaxis:#3)}] (0,0) -- (30:1) -- (-30:1) --(0,-1)--(0,0);
}

% The cube 
\newcommand\cube[3]{
	\topside{#1}{#2}{#3} \leftside{#1}{#2}{#3} \rightside{#1}{#2}{#3}
}

\newcommand\planepartition[1]{
	\setcounter{x}{-1}
	\foreach \a in {#1} {
		\addtocounter{x}{1}
		\setcounter{y}{-1}
		\foreach \b in \a {
			\addtocounter{y}{1}
			\setcounter{z}{-1}
			\foreach \c in {1,...,\b} {
				\addtocounter{z}{1}
				\cube{\value{x}}{\value{y}}{\value{z}}
			}
		}
	}
}

%! coverlogo

\makeatletter
\pgfdeclareplotmark{dot}
{%
	\fill circle [x radius=0.02, y radius=0.08];
}%
\makeatother

\begin{document}
	\thispagestyle{empty}
	\vspace{30em}
	\begingroup  
	\centering
	\textsc{\LARGE Lineáris regresszió elméleti összefoglaló}\\[2.5em]
	\small{2023.03.19}\\
	\large Bognár Miklós\par
	\endgroup
	\vspace{10em}
	\begin{center}
	\begin{tikzpicture}[scale=0.9]
		%\planepartition{{5,5,5,5,5},{0},{3,3,3},{},{1}}
		\draw[step=1cm,pink,very thin] (-1.9,-1.9) grid (5.9,5.9);
		\draw[thick,->] (0,0) -- (5,0);
		\draw[thick,->] (0,0) -- (0,5);
		\draw[dashed,->] (0,1) -- (4,4);
		\node at (1,3.5) [circle,fill,inner sep=1.5pt]{};
		\node at (2,1.5) [circle,fill,inner sep=1.5pt]{};
		\node at (3,4) [circle,fill,inner sep=1.5pt]{};
		\node at (3.3,2) [circle,fill,inner sep=1.5pt]{};
		\node at (2.1,3.2) [circle,fill,inner sep=1.5pt]{};
		\node at (1.6,1.6) [circle,fill,inner sep=1.5pt]{};
		\node at (1.8,1.8) [circle,fill,inner sep=1.5pt]{};
	\end{tikzpicture}
	\end{center}
	
	\tableofcontents
	\vspace{2cm}
	\emph{A *-al jelölt fejezetek/alfejezetek tudtommal nem képezik részét az anyagnak, azonban (szerintem) érdekesek, és segíthetnek jobban megérteni a lineáris regressziót.}
	\newpage
	\chapter{Matematikai összefoglaló}
	A lineáris regresszió megértéséhez elengedhetetlen, hogy tisztában legyünk néhány, lineáris algebrából ismeretes fogalommal és összefüggéssel. Ezen felül nagyon hasznos, ha ismerjük, hogy hogyan kezelendőek a valószínűségi vektorváltozók illetve a mártixdifferenciálás-kifejezések.  
	
	\section{Pszeudoinverzek}
	Legyen $\pmb{A} \in \mathbb{R}^{n\times m}, n \ne m$ nem négyzetes mátrix. Ha egy  $\pmb{A}x = y , x \in \mathbb{R}^{m\times 1}, y \in \mathbb{R}^{n \times 1}$ lineáris egyenletrendszer együtthatómátrixaként gondolunk rá, akkor $n > m$ vagy $m > n$ esetén rendre a \emph{túlhatározottság} vagy \emph{alulhatározottság} esete állna fent, az első esetben általánosságban nem lenne megoldásunk, a második esetben pedig végtelen sok megoldásunk lenne rá. Látszik, hogy az $n \ne m$ esetben nem beszélhetünk $\pmb{A}^{-1}$ inverzről, helyette egy általánosabb, úgynevezett \emph{pszeudoinverz} kell.
	\\
	\\
	Egy $\pmb{A} \in \mathbb{R}^{n \times m}, n > m$ mátrix \emph{bal oldali pszeudoinverze} (Más néven \emph{Moore-Penrose pszeudoinverz}):
	\[
	\pmb{A}^{\dagger} := (\pmb{A}^T\pmb{A})^{-1}\pmb{A}^T \in \mathbb{R}^{m \times n}
	\]
	Figyeljük meg, hogy ha $\pmb{A}^{\dagger}$-el balról megszorozzuk $\pmb{A}$-t, az identitás mátrixot kapjuk, tehát bal oldalról valóban identitásként működik:
	\[
	\pmb{A}^{\dagger}\pmb{A} = (\pmb{A}^T\pmb{A})^{-1}\pmb{A}^T\pmb{A} = \pmb{I}
	\]
	Ha jobbról szoroznánk meg:
	\[
	\pmb{A}\pmb{A}^{\dagger} = \pmb{A}(\pmb{A}^T\pmb{A})^{-1}\pmb{A}^T
	\]
	Ez semmi más, mint a \emph{projekció-mátrix} $\pmb{A}$ oszlopvektorai által kifeszített vektortérre. Ha egy vektor ebben az oszloptérben van, rá persze identitásként hat $\pmb{A}\pmb{A}^{\dagger}$, ha viszont ezen kívül esik, akkor rávetíti az oszloptérre a vektort. Egy túlhatározott $\pmb{A}x = y$ egyenletrendszert tehát "meg lehet oldani", ha $y$-t rávetítjük $\pmb{A}$ oszlopterére, és megoldjuk az $\pmb{A}x = \tilde{y}$ egyenletrendszert: 
	\[
	\tilde{y} = \pmb{A}(\pmb{A}^T\pmb{A})^{-1}\pmb{A}^T y = \pmb{A}\pmb{A}^{\dagger} y
	\]
	\[
	\pmb{A}x = \tilde{y} = \pmb{A}\pmb{A}^{\dagger}y
	\]
	\[
	\pmb{A}^{\dagger}\pmb{A}x = \pmb{A}^{\dagger}\pmb{A}\pmb{A}^{\dagger}y
	\]
	\[
	x  = \pmb{A}^{\dagger}y
	\]
	Az $n < m$ esetben alulhatározottság áll fenn, itt \emph{jobb oldali pszeudoinverzről} beszélhetünk:
	\[
	\pmb{A}^{\ddagger} := \pmb{A}^T(\pmb{A}\pmb{A}^T)^{-1} \in \mathbb{R}^{n \times m}
	\]
	Bár ezt nem fogjuk a későbbiekben használni, érdemes lehet megjegyezni, hogy a jobb oldali pszeudoinverzzel való balról szorzás esetén - hasonlóan a bal oldali pszeudoinverzhez - projekciómátrixot kapunk, csak most $\pmb{A}$ sorvektorai által kifeszített vektortérre. Bal oldali pszeudoinverz csakis $n \ge m$ esetben létezik, míg jobboldali az $n \le m$ esetben.
	
	\section{*Mátrixok szinguláris értékei, SVD}
	Legyen $\pmb{A} \in \mathbb{C}^{n \times m}$ tetszőleges komplex mátrix. Ekkor $\pmb{A}$ \emph{szinguláris érték felbontása (Singular Value Decomposition - SVD)}:
	\[
		\pmb{A} = \pmb{U}\pmb{S}\pmb{V}^*
	\]
	ahol $\pmb{U} \in \mathbb{C}^{n \times n}$ és $\pmb{V} \in \mathbb{C}^{m \times m}$ unitér mátrixok, és $\pmb{S} \in \mathbb{R}^{n \times m}$ kvázi-diagonális, azaz $n > m$ esetben
	\[
		\pmb{S} =
		\begin{bmatrix}
		\sigma_1 & 0 & \dots & 0 \\
		0 & \sigma_2 & \dots & 0 \\
		\vdots & \vdots & \ddots & \vdots \\
		0 & 0 & \dots & \sigma_m \\
		0 & 0 & \dots & 0 \\
		\vdots & \vdots & \ddots & \vdots \\
		0 & 0 & \dots & 0
		\end{bmatrix}_{n \times m}
	\]
	Ilyen felbontás \emph{mindig} létezik, bármilyen is legyen $\pmb{A}$ dimenziója. Ha $\pmb{A}$ valós mátrix, akkor $\pmb{U}$ és $\pmb{V}$ \emph{ortogonálisak}, és így persze a konjugált transzponálás ekvivalens lesz a transzponálással. $\pmb{S}$ $\sigma$ elemei a \emph{szinguláris értékei} $\pmb{A}$-nak. Figyeljük meg, hogy
	\[
		\pmb{A}^T\pmb{A} = \pmb{V}\pmb{S}^T\pmb{U}^T\pmb{U}\pmb{S}\pmb{V}^T = \pmb{V}\pmb{S}^T\pmb{S}\pmb{V}^T,
	\]
	azaz $\pmb{A}^T\pmb{A}$ spektrálfelbontása lesz. $\pmb{V}$ oszlopai tehát $\pmb{A}^T\pmb{A}$ sajátvektorai lesznek, míg hasonlóan belátható, hogy $\pmb{U}$ oszlopai pedig $\pmb{A}\pmb{A}^T$ sajátvektorai lesznek (gondoljuk meg, hogy minden szimmetrikus valós mátrix ortogonálisan spektrálfelbontható). Mindkét esetben $\pmb{S}^T\pmb{S}$ négyzetes mátrix diagonális elemei a szinguláris értékek négyzetei lesznek, azaz kimondható, hogy $\pmb{A}$ szinguláris értékei semmi mások, mint $\pmb{A}^T\pmb{A}$ sajátértékeinek négyzetgyökei. Innen persze az is következik, hogy ha $\pmb{A}^T\pmb{A}$ szinguláris, azaz van $0$ sajátértéke, akkor biztosan lesz $0$ szinguláris értéke $\pmb{A}$-nak. Innen következik, hogy ha még mindig az $n > m$ esetnél maradva $\pmb{A}$ oszloprangja kisebb, mint oszlopainak száma (lineárisan összefüggő oszlopai vannak), akkor $\pmb{A}^T\pmb{A}$-nak lesz $0$ sajátértéke, tehát nem lesz invertálható.
	\\
	\\
	Az SVD segítségével kifejezhető $\pmb{A}$ Moore-Penrose pszeudoinverze is (az inverz a transzpozícióhoz hasonlóan megfordítja a mátrixszorzás sorrendjét):
	\[
		\pmb{A}^{\dagger} = (\pmb{A}^T\pmb{A})^{-1}\pmb{A}^T = (\pmb{V}\pmb{S}^T\pmb{U}^T\pmb{U}\pmb{S}\pmb{V}^T)^{-1}\pmb{V}\pmb{S}^T\pmb{U}^T = \pmb{V}\pmb{S}^{-1}{\pmb{S}^T}^{-1}\pmb{V}^T\pmb{V}\pmb{S}^T\pmb{U}^T
	\]
	\[
		\pmb{A}^{\dagger} = \pmb{V}\pmb{S}^{\dagger}\pmb{U}^T
	\]
	Mivel $\pmb{S}$ maga sem négyzetes mátrix feltétlenül, így $\pmb{S}^{-1}$ és ${\pmb{S}^T}^{-1}$ valójában $\pmb{S}^{\dagger}$ illetve ${\pmb{S}^T}^{\dagger}$ Moore-Penrose pszeudoinverzeket jelenti. A pszeudoinverz tulajdonságai hasonlók az egyszerű inverzéhez.
	Innen is látszik, hogy $\pmb{A}^{\dagger}$ csak akkor létezik, ha $\pmb{S}^{\dagger}$ létezik, ami persze $\pmb{S}$ kvázi-diagonalitásából következően akkor igaz, ha $\pmb{S}$ oszlopai között nincs csupa $0$-ákból álló, azaz nincs $\sigma = 0$ szinguláris értéke $\pmb{A}$-nak.
	
	\section{Valószínűségi vektorváltozók}
	Egy $\pmb{\xi} = [\xi_1, \dots , \xi_n]^T$ vektort \emph{valószínűségi vektorváltozónak} hívunk, ha $\forall i$-re $\xi_i$ skalárértékű valószínűségi változó. A továbbiakban csak a vektorértékű normális eloszlást követő valószínűségi vektorváltozókkal foglalkozunk, ezek formálisan felírva:
	\[
	\pmb{\xi} \sim \mathcal{N}(\pmb{\mu}, \pmb{\Sigma})
	\]
	ahol $\pmb{\mu} \in \mathbb{R}^{n \times 1}$ a várható értékek vektora, $\pmb{\Sigma}$ pedig a \emph{variancia-kovarianca mátrix}. Természetesen $Var[\pmb{\xi}] = \pmb{\Sigma}$. Természetesen $\pmb{\Sigma} \in \mathbb{R}^{n \times n}$ pozitív szemidefinit és szimmetrikus mártix. Az $n = 1$ esettel analóg módon $\pmb{\xi}$ sűrűségfüggvénye ($\xi_i$-k most konkrét értékek)
	\[
	f_{\pmb{\xi}}(\xi_1, \dots, \xi_n) = \frac{ e^{-\frac{1}{2}(\pmb{\xi}-\pmb{\mu})^T\pmb{\Sigma}^{-1}(\pmb{\xi}-\pmb{\mu})}}{\sqrt{(2\pi)^n|\pmb{\Sigma}|}}
	\]
	A sűrűségfüggvény $n = 2$ esetben $\pmb{\mu} = [2, -1]^T$ és $\pmb{\Sigma} = \pmb{I}$ várhatóérték és kovariancia mátrix mellett: 
	\def\centerx{2}
	\def\centery{-1}
	
	\begin{center}
		\begin{tikzpicture}
			\begin{axis}
				\addplot3[surf,domain=-2:6,domain y=-5:3] 
				{exp(-( (x-\centerx)^2 + (y-\centery)^2)/3 )};
				\node[circle,inner sep=1pt,fill=blue,pin=90:$\pmb{\mu}$] 
				at (axis cs:\centerx,\centery,1) {};
			\end{axis}
		\end{tikzpicture}
	\end{center} 
	Egy $\pmb{A} \in \mathbb{R}^{n \times n}$ mátrix mellett a skaláresethez hasonlóan 
	\[
	Var[\pmb{A}\pmb{\xi}] = \pmb{A} \pmb{\Sigma} \pmb{A}^T
	\]
	\[
	\mathbb{E}[\pmb{A}\pmb{\xi}] = \pmb{A}\mathbb{E}[\pmb{\xi}]
	\]
	$\pmb{\Sigma}$ kovariancia mátrixot kifejezhetjük várható értékekkel is:
	\[
	\pmb{\Sigma} = \mathbb{E}[(\pmb{\xi}-\mathbb{E}[\pmb{\xi}])(\pmb{\xi}-\mathbb{E}[\pmb{\xi}])^T] = \mathbb{E}[\pmb{\xi}\pmb{\xi}^T] - \mathbb{E}[\pmb{\xi}]\mathbb{E}[\pmb{\xi}^T]
	\]
	$\pmb{\Sigma}$ alakja:
	\[
	\pmb{\Sigma} = 
	\begin{bmatrix}
		\sigma_1^2 & Cov[\xi_1,\xi_2] & \dots & Cov[\xi_1,\xi_n] \\
		Cov[\xi_2,\xi_1] & \sigma_2^2 & \dots & Cov[\xi_2,\xi_n] \\
		\vdots & \vdots & \ddots & \vdots \\
		Cov[\xi_n,\xi_1] & Cov[\xi_n,\xi_2] & \dots & \sigma_n^2
	\end{bmatrix}
	\]
	ahol $\sigma_1^2, \dots, \sigma_n^2$ rendre $\xi_1, \dots, \xi_n$ varianciái. 
	
	\section{Mátrixdifferenciálás nagyon röviden}
	Legyenek $\pmb{a}, \pmb{b} \in \mathbb{R}^{k \times 1}$ vektorok. Ekkor
	\[
	\frac{\partial\pmb{a}^T\pmb{b}}{\partial\pmb{b}} = \frac{\partial\pmb{b}^T\pmb{a}}{\partial\pmb{b}} = \pmb{a}
	\]
	Ha $\pmb{A} \in \mathbb{R}^{k \times k}$ mátrix, akkor
	\[
	\frac{\partial\pmb{b}^T\pmb{A}\pmb{b}}{\partial\pmb{b}} = 2\pmb{A}\pmb{b}
	\]
	Ha $\pmb{A}$ szimmetrikus, akkor ezen felül
	\[
	2\pmb{A}\pmb{b} = 2\pmb{b}^T\pmb{A}
	\]
	Legyen $\pmb{\beta} \in \mathbb{R}^{k \times 1}$, $\pmb{A} \in \mathbb{R}^{n \times k}$ és $\pmb{y} \in \mathbb{R}^{n \times 1}$. Ekkor
	\[
	\frac{\partial 2\pmb{\beta}^T\pmb{A}^T\pmb{y}}{\partial\pmb{\beta}} = \frac{\partial 2\pmb{\beta}^T(\pmb{A}^T\pmb{y})}{\partial\pmb{\beta}} = 2\pmb{A}^T\pmb{y}
	\]
	
	\section{*Pont és eloszlás Mahalanobis távolsága}
	Ez a rész csak érdekességként szerepel a PDF-ben, a Generalized Least Squares paraméterbecslés analitikus levezetésének bemutatásában használjuk csak, akinek nincs ideje átolvasni ezt a részt, nyugodtan ugorja át.
	\\
	\\
	Legyen $F$ egy $\mathbb{R}^{n}$-en értelmezett eloszlás $\pmb{\mu} = [\mu_1, \mu_2, \dots, \mu_n]^T$ várható értékekkel és egy pozitív definit $\pmb{\Sigma}$ variancia-kovariancia mátrixxal. Egy $\pmb{x} = [x_1, x_2, \dots, x_n]^T$ pont Mahalanobis távolsága $F$-től
	\[
		d_M(\pmb{x}, F) := \sqrt{(\pmb{x} - \pmb{\mu})^T\pmb{\Sigma}^{-1}(\pmb{x} - \pmb{\mu})}
	\]
	Kettő $\pmb{x}, \pmb{y} \in \mathbb{R}^n$ pont $F$ szerinti Mahalanobis távolsága:
	\[
		d_M(\pmb{x}, \pmb{y}; F) := \sqrt{(\pmb{x} - \pmb{y})^T\pmb{\Sigma}^{-1}(\pmb{x} - \pmb{y})}
	\]
	\section{*Particionált mátrixok és a Blockwise formula}
	Legyen $\pmb{A} \in \mathbb{R}^{n \times n}$ négyzetes invertálható mátrix. Particionáljuk $\pmb{A}$-t az alábbi módon:
	\[
		\pmb{A} =
		\begin{bmatrix}
		\pmb{A}_{11} & \pmb{A}_{12} \\
		\pmb{A}_{21} & \pmb{A}_{22}
		\end{bmatrix}
	\]
	ahol $\pmb{A}_{11} \in \mathbb{R}^{m_1 \times m_1}$, $\pmb{A}_{22} \in \mathbb{R}^{m_2 \times m_2}$, $m_1 + m_2 = m$ maguk is invertálható mátrixok. Ekkor $\pmb{A}^{-1}$ felírható:
	\[
		\pmb{A}^{-1} = 
		\begin{bmatrix}
		(\pmb{A}_{11}-\pmb{A}_{12}\pmb{A}_{22}^{-1}\pmb{A}_{21})^{-1} & -\pmb{A}_{11}^{-1}\pmb{A}_{12}(\pmb{A}_{22}-\pmb{A}_{21}\pmb{A}_{11}^{-1}\pmb{A}_{12})^{-1} \\
		-\pmb{A}_{22}^{-1}\pmb{A}_{21}(\pmb{A}_{11}-\pmb{A}_{12}\pmb{A}_{22}^{-1}\pmb{A}_{21})^{-1} & (\pmb{A}_{22}-\pmb{A}_{21}\pmb{A}_{11}^{-1}\pmb{A}_{12})^{-1}
		\end{bmatrix}
	\]
	A négyzetes mátrixokról az általános $n \times m$-es mátrixokra áttérve legyen $\pmb{X} \in \mathbb{R}^{n \times m}$ tetszőleges valós mátrix, $n > m$, és particionálja ezt horizontálisan $\pmb{Q}$ és $\pmb{R}$:
	\[
		\pmb{X} = \begin{bmatrix}
				\pmb{Q} & \pmb{R}
		\end{bmatrix}
	\]
	Ekkor $\pmb{X}\pmb{X}^{\dagger}$ oszloptér-vetítés mátrix a következőképpen írható fel:
	\[
		\pmb{X}\pmb{X}^{\dagger} = \pmb{Q}\pmb{Q}^{\dagger} + ((\pmb{I}-\pmb{Q}\pmb{Q}^{\dagger})\pmb{R})((\pmb{I}-\pmb{Q}\pmb{Q}^{\dagger})\pmb{R})^{\dagger}
	\]
	Ez az úgynevezett \emph{Blockwise formula}, avagy a blokkonkénti vetítés formula. Még mindig ennél a partíciónál maradva $\pmb{X}^{\dagger}$-t is megkaphatjuk a blokkokkal:
	\[
		\pmb{X}^{\dagger} = 
		\begin{bmatrix}
		\pmb{Q} & \pmb{R}
		\end{bmatrix}^{\dagger} =
	\begin{bmatrix}
	\pmb{P}_{R}\pmb{Q}(\pmb{Q}^T\pmb{P}_{R}\pmb{Q})^{-1} \\
	\pmb{P}_{Q}\pmb{R}(\pmb{R}^T\pmb{P}_{Q}\pmb{R})^{-1}
	\end{bmatrix} = 
	\begin{bmatrix}
	(\pmb{P}_{R}\pmb{Q})^{\dagger} \\
	(\pmb{P}_{Q}\pmb{R})^{\dagger}
	\end{bmatrix}
	\]
	ahol $\pmb{P}_{Q}$ és $\pmb{P}_{R}$ rendre az ortogonális projekciómátrixok a $\pmb{Q}$ és $\pmb{R}$ mátrixok képterére ortogonális vektortérre:
	\[
		\pmb{P}_{Q} = \pmb{I} - \pmb{Q}(\pmb{Q}^T\pmb{Q})^{-1}\pmb{Q}^T = \pmb{I} - \pmb{Q}\pmb{Q}^{\dagger}
	\]
	\[
		\pmb{P}_{R} = \pmb{I} - \pmb{R}(\pmb{R}^T\pmb{R})^{-1}\pmb{R}^T = \pmb{I} - \pmb{R}\pmb{R}^{\dagger}
	\]
	Tehát
	\[
		\begin{bmatrix}
			\pmb{Q} & \pmb{R}
		\end{bmatrix}^{\dagger} =
		\begin{bmatrix}
			((\pmb{I}-\pmb{R}\pmb{R}^{\dagger})\pmb{Q})^{\dagger} \\
			((\pmb{I}-\pmb{Q}\pmb{Q}^{\dagger})\pmb{R})^{\dagger}
		\end{bmatrix}
	\]
	Ezek a formulák akkor lehetnek hasznosak, ha a lineáris regresszió \emph{design mátrixát} particionáljuk bizonyos magyarázó változók szerint (például $\pmb{Q}$ lehet a csupa 1-ekből álló intercept oszlopmátrix), de ez már nagyon túlmutat a tárgy anyagán.  
	
	\chapter{A lineáris regresszió}
	\section{A regressziós modell}
	A regresszió kiindulópontja egy $\mathscr{X}$ sokaság, melynek minden tagja rendelkezik $\pmb{x}_i$ \emph{featurevektor}-ral, avagy magyarázó változó-vektorral (ezek a \emph{regresszorok}), illetve egy-egy skalár $y_i$ \emph{label}-lel, avagy magyarázott változóval (amiket a regresszorok magyaráznak egy lineáris modell alapján, ezt később jobban kifejtjük). A sokaságból $n$ darab mintát veszünk (megfigyelést végzünk), \emph{a minták iid-k, azaz függetlenek és azonos eloszlásúak}, ami persze azt jelenti, hogy \emph{minden magyarázó változó-vektor egy vektorértékű valószínűségi vektorváltozó}. Létezik egy másik konstrukció is, miszerint $\pmb{X}$ rögzített, és nem változik mintavételről mintavételre, ez azonban csak annyit jelent, hogy mindenhol, ahol feltételes eloszlás/várható érték van, onnan az $\mid \pmb{X}$ feltételt ki kell venni. Mi $\pmb{X}$-re mint valószínűségi vektorváltozók mátrixa tekintünk mostantól.
	\\
	\\
	A megfigyelt magyarázó változó-vektorokat soronként egymásra rakva felépítünk egy úgynevezett \emph{design mátrix}ot, melyet mostantól $\pmb{X}$-el jelölünk. Minden $\pmb{x}_i$ magyarázó változó-vektor első eleme konstans $1$, ez tölti be az intercept, avagy kétdimenziós esetben az y-tengellyel való metszéspont szerepét. $n$ darab megfigyelés és $p$ elemszámú magyarázó változó-vektorral $\pmb{X}$ alakja a következő:
	\[
	\pmb{X} = 
	\begin{bmatrix}
		1 & x_{1,1} & \dots & x_{1,p-1} \\
		1 & x_{2,1} & \dots & x_{2,p-1} \\
		\vdots & \vdots & \ddots & \vdots \\
		1 & x_{n,1} & \dots & x_{n,p-1}	
	\end{bmatrix}_{n \times p}
	\]
	A megfigyelt magyarázott változókat szintén sorokba tömörítjük, így mivel mindegyik skalár, egy vektort kapunk:
	\[
	\pmb{y} =
	\begin{bmatrix}
		y_1 \\
		y_2 \\
		\vdots \\
		y_n
	\end{bmatrix}
	\]
	A lineáris regresszió kiindulópontja mindig egy \emph{modell}, avagy egy elméleti feltevés arról, hogy milyen kapcsolatban áll a magyarázott $\pmb{y}$ változó a magyarázó $\pmb{X}$ regresszorokkal.
	\[
	\pmb{y} = \pmb{X}\pmb{\beta} + \pmb{\epsilon}
	\]
	A lineáris kapcsolatot a $\pmb{\beta}$ együtthatóvektor (avagy \emph{paramétervektor}) írja le, míg $\pmb{\e}$ a regresszorok által nem magyarázott eltéréseket, avagy \emph{hibákat} jelenti. Mostantól $\pmb{\e}$-re \emph{hibavektor} néven hivatkozunk.
	\\
	\\
	A regresszió célja, hogy megtaláljuk azt a $\ebeta$ \emph{paraméterbecslés-vektort}, hogy az 
	\[
		\hat{\pmb{y}} = \pmb{X}\ebeta 
	\]
	úgynevezett \emph{predikciós} egyenletből származott \emph{becsült} $\hat{\pmb{y}}$ vektor a lehető legközelebb legyen a valódi megfigyelt $\pmb{y}$ vektorhoz. Persze megfigyeletlen $\pmb{x}$ magyarázó változók esetén a predikciós egyenlet szintén működik, és valójában ez is a célja a regressziónak.
	\\
	\\
	A lineáris regresszió egy darab regresszor (magyarázó változó) esetén az alábbi ábrával szemléltethető:
	
	\begin{center}
		\begin{tikzpicture}
			
			\begin{axis}[
				xmin = 0, xmax = 11,
				ymin = 0, ymax = 11,
				width = 0.5\textwidth,
				height = 0.35\textwidth,
				xtick distance = 1,
				ytick distance = 1,
				xlabel={$x$},
				ylabel={$y$},
				grid = both,
				minor tick num = 1,
				major grid style = {lightgray},
				minor grid style = {lightgray!25},
				]
				\addplot[teal, only marks] table[x = t, y = x] {regr-dat.dat};
				
				\addplot[
				thick,
				orange
				] table[
				x = t,
				y = {create col/linear regression={y=x}}
				] {regr-dat.dat};
				
				\addlegendentry{Adatpontok}
				\addlegendentry{
					Regresszió: $ \hat{y} =
					\pgfmathprintnumber{\pgfplotstableregressiona}
					\cdot x
					\pgfmathprintnumber[print sign]{\pgfplotstableregressionb}$
				};
				
			\end{axis}
		\end{tikzpicture}
	\end{center}
	Itt $\ebeta$ paraméterbecslés vektor alakja
	\[
	\ebeta = 
	\begin{bmatrix}
		\hat{\beta_0} \\
		\hat{\beta_1}
	\end{bmatrix}
	=
	\begin{bmatrix}
		1,87 \\
		0,69
	\end{bmatrix}
	\]
	Azt, hogy hogyan kaptuk meg $\ebeta$ paraméterbecslést, a következő fejezetek tárgyalják részletesen. Ezen kívül külön foglalkozunk majd a fenti egyváltozós regresszióval is (a $p=2$-es eset).
	
	\section{Az Ordinary Least Squares (OLS) becslési eljárás}
	A lineáris regresszió $\ebeta$-jának megtalálására az egyik lehetséges eljárás az Ordinary Least Squares, avagy legkisebb négyzetek módszere. Az eljárást kettő szemszögből is megvizsgáljuk.
	
	\subsection{Az OLS-becslés geometriai értelmezése}
	Szinte mindig $n > p$, azaz több megfigyelésünk van, mint amennyi magyarázó változónk, így az
	\[
		\pmb{X}\pmb{\beta} = \pmb{y}
	\]
	 egyenletrendszer \emph{túlhatározott}, és nagyon specifikus esetektől eltekintve nem létezik egzakt megoldás $\pmb{\beta}$-ra. Az első fejezetben azonban láttuk, hogy a bal oldali pszeudoinverz pontosan ezt a problémát orvosolja. A jelölési konvenció a megoldásból nyert \emph{paraméter-becslésre} $\ebeta$, ami a mintavétel véletlenszerűségéből adódóan maga is vektorértékű valószínűségi változó ($\ebeta$ pontos eloszlásáról a későbbiekben lesz szó):
	\[
	\ebeta = \pmb{X}^{\dagger}\pmb{y} = (\pmb{X}^T\pmb{X})^{-1}\pmb{X}^T\pmb{y}
	\]
	
	\begin{center}
	\begin{tikzpicture}
		\begin{axis}[hide axis]
			\addplot3 [
			domain=0:5,
			domain y = 0:5,
			samples = 40,
			samples y = 40,
			surf] {x + y};
			
			\node[] at (axis cs: 4,2) {$\pmb{X}$ oszloptere};
			
			%\draw[|->, black](0,0,0) -- (1,1,10);
			\coordinate (A) at (0,0,0);
			\coordinate (B) at (1,3,0);
			\draw [stealth-stealth] (A) -- (B) node[below,font=\small]{};
			\draw [ -stealth] (0,0,0) -- (1,1,9) coordinate (yn) node[right]{$y$};
			\draw[dashed] (yn) --  node[midway,above right]{$e$} ($(A)!(yn)!(B)$) node[above left]{};
		\end{axis}
		
	\end{tikzpicture}
	\end{center}
	Ebben az esetben $\pmb{y}$-t az $\pmb{X}$ design mátrix oszlopterére vetítettük. $\pmb{X}^T\pmb{X}$ \emph{Gram-mátrix} néven is ismeretes (egyébként $\pmb{X}\pmb{X}^T$-ra is szoktak utalni ezen a néven, annyi különbséggel, hogy az előbbi a regresszorok közti korreláció mértékét mutatja a mintavételeken keresztülfutva, egyfajta \emph{temporális} módon, az utóbbi pedig magukon a regresszorokon keresztülfutva egyfajta \emph{térbeli} korrelációt mutat). Az $\pmb{X}^T\pmb{X}$ mátrix determinánsát \emph{Gram-determinánsnak} is hívják.
	\\
	\\
	Ha $n < p$, azaz kevesebb megfigyelésünk van, mint amennyi magyarázó változónk, az egyenletrendszer alulhatározott lesz, és nem fog létezni bal oldali pszeudoinverz, így nem lesz olyan $\pmb{X}^{\dagger}$ mátrix, amivel balról beszorozva $\pmb{X}$-et az identitásmátrixot kapnánk. Ha $\pmb{X}^{\ddagger}$-el próbálkozunk, ami létezik:
	\[
		\pmb{X}^{\ddagger}\pmb{X}\beta = \pmb{X}^{\ddagger}\pmb{y}
	\] 
	a bal oldalon $\pmb{X}$ sorterére való vetítési mátrixot kapnánk. Innen az is következik, hogy amint megtaláltuk $\ebeta$ első $n$ elemét, a maradék $p-n$ együttható az első $n$ együttható lineáris kombinációjaként állna elő szükségszerűen. Ezért mostantól feltesszük, hogy a "normális" $n > p$ eset áll fenn. 
	\\
	\\
	A továbbiakban a \emph{tényleges hibavektor} jelölése 
	$\pmb{e}$, a valós $y_i$-k és a $\pmb{X}\ebeta = \hat{\pmb{y}}$ modellbecslés által prediktált $\hat{y_i}$-k közti eltérések vektora (sokszor $\pmb{e}$-t $\widehat{\pmb{\e}}$-ként is jelölik):
	\[
	\pmb{e} =
	\begin{bmatrix}
		y_1 - \hat{y_1} \\
		y_2  - \hat{y_2} \\
		\vdots \\
		y_n - \hat{y_n}
	\end{bmatrix}
	\]
	\subsection{Az OLS-becslés mint szélsőérték-feladat}
	$\ebeta$ paraméterbecslés-vektort megkaphatjuk úgy is, ha tekintjük az alábbi minimalizálási feladatot:
	\[
	\pmb{e}^T\pmb{e} \rightarrow \min_{\ebeta}
	\]
	azaz minimalizáljuk a becsült $\hat{y}_i$ és tényleges $y_i$ magyarázott változók közötti négyzetösszeget. $\pmb{e}^T\pmb{e}$-t RSS, azaz \emph{sum of squared residuals} néven is emlegetik. Írjuk ki a hiba-négyzetösszeg teljes alakját:
	\[
	\pmb{e}^T\pmb{e} = (\pmb{y} - \pmb{X}\ebeta)^T(\pmb{y} - \pmb{X}\ebeta) = \pmb{y}^T\pmb{y} - \ebeta^T\pmb{X}^T\pmb{y} - \pmb{y}^T\pmb{X}\ebeta + \ebeta^T\pmb{X}^T\pmb{X}\ebeta = \pmb{y}^T\pmb{y} - 2\ebeta^T\pmb{X}^T\pmb{y} + \ebeta^T\pmb{X}^T\pmb{X}\ebeta
	\]
	Itt felhasználtuk, hogy a transzponálás "megfordítja a szorzatot", illetve hogy skalár transzponáltja önmaga, így $\pmb{y}^T\pmb{X}\ebeta = (\pmb{y}^T\pmb{X}\ebeta)^T = \ebeta^T\pmb{X}^T\pmb{y}$.
	A minimalizációhoz vennünk kell a kifejezés $\ebeta$ szerinti deriváltját, majd $0$-val egyenlővé tenni:
	\[
	\frac{\partial{\pmb{e}^T\pmb{e}}}{\partial\ebeta} = -2\pmb{X}^T\pmb{y} + 2\pmb{X}^T\pmb{X}\ebeta = 0
	\]
	Ebből megkapjuk az úgynevezett \emph{normálegyenletet}:
	\[
	(\pmb{X}^T\pmb{X})\ebeta = \pmb{X}^T\pmb{y}
	\]
	$(\pmb{X}^T\pmb{X})$ szimmetrikus, és ha feltesszük, hogy létezik inverze, akkor balról beszorozva mindét oldalt:
	\[
	(\pmb{X}^T\pmb{X})^{-1}(\pmb{X}^T\pmb{X})\ebeta = (\pmb{X}^T\pmb{X})^{-1}\pmb{X}^T\pmb{y}
	\]
	\[
	\ebeta = (\pmb{X}^T\pmb{X})^{-1}\pmb{X}^T\pmb{y}
	\]
	Látható, hogy pontosan ugyanaz jött ki, mint a pszeudoinverzes levezetésben. Míg ez utóbbi pusztán analitikus úton jutott el $\ebeta$-hoz, a pszeudoinverzes módszert geometrikus úton is el lehet képzelni.
	
	\subsection{Az OLS-becslés tulajdonságai}
	Vegyük az OLS paraméterbecslés normálegyenletét, és figyeljük meg, hogy $\pmb{X}^T\pmb{e} = \pmb{0}$:
	\[
	(\pmb{X}^T\pmb{X})\ebeta = \pmb{X}^T\pmb{y}
	\]
	A modellből adódóan $\pmb{y} = \pmb{X}\ebeta + \pmb{e}$ behelyettesítéssel:
	\[
	(\pmb{X}^T\pmb{X})\ebeta = \pmb{X}^T(\pmb{X}\ebeta + \pmb{e})
	\]
	\[
	(\pmb{X}^T\pmb{X})\ebeta = (\pmb{X}^T\pmb{X})\ebeta + \pmb{X}^T\pmb{e}
	\]
	\[
	\pmb{X}^T\pmb{e} = \pmb{0}
	\]
	valóban. Ez azt jelenti, hogy \emph{minden magyarázó változó (regresszor) korrelálatlan a hibával}, pontosabban megfogalmazva \emph{a regresszorok és a hibák mintakorrelációja zérus}. Mivel $\pmb{X}$ mátrix első oszlopa konstans $1$-eket tartalmaz, így $\hat{\beta}_0$ maga az intercept lesz, és emiatt 
	\[
	\sum_{i=1}^{n}e_i = 0
	\]
	azaz a hibák összege $0$. Ha leosztunk $n$-nel:
	\[
	\frac{1}{n}\sum_{i=1}^{n}e_i = \bar{\pmb{e}}
	\]
	azaz a hibatagok (\emph{rezidiumok}) mintaátlaga - ami persze torzítatlan becslése a várható értéknek - $0$, tehát $\mathbb{E}[\pmb{e}] = \pmb{0}$.
	\\
	\\
	Egy másik, ugyancsak fontos tulajdonság a predikciós formulából következik:
	\[
	\hat{\pmb{y}}^T\pmb{e} = (\pmb{X}\ebeta)^T\pmb{e} = \ebeta^T\pmb{X}^T\pmb{e} = 0
	\]
	azaz \emph{a becsült $\hat{y}_i$-ok korrelálatlanok a rezidiumokkal}. Így azt is beláthatjuk, hogy \emph{a modell által prediktált és a tényleges magyarázott változók mintaátlagai megegyeznek}:
	\[
	\bar{\pmb{y}} = \bar{\hat{\pmb{y}}}
	\]
	Felmerülhet a kérdés, hogy mindig létezik-e $(\pmb{X}^T\pmb{X})^{-1}$. Abban az esetben, ha $\pmb{X}$ oszloprangja kisebb, mint $p$, tehát \emph{tökéletes multikollinearitás} áll fenn, akkor $\pmb{X}$ szinguláris értékei között lesz $0$, így $\pmb{X}^T\pmb{X}$ sajátértékei között is, azaz $\pmb{X}^T\pmb{X}$ nem lesz invertálható. Ezentúl tehát feltételezzük, hogy nem áll fenn tökéletes multikollinearitás.
	
	\section{A Gauss-Markov feltételezések}
	A Gauss-Markov feltételezések biztosítják, hogy a \emph{Gauss-Markov tétel} értelmében az OLS eljárással kapott $\ebeta$ paraméterbecslésünk \emph{BLUE}, azaz \emph{Best Linear Unbiased Estimator} lesz. Ez azt jelenti, hogy nem fogunk tudni találni olyan - nem az OLS eljárással kapott - paraméterbecslést $\pmb{\beta}$-ra, ami lineáris, torzítatlan, és kisebb mintavarianciával rendelkezne, mint $\ebeta$ (az utóbbi tulajdonságra mint $\ebeta$ \emph{hatásossága} szoktak hivatkozni).
	\\
	\\ 
	Formálisan kimondva az első Gauss-Markov feltétel a már látott modellegyenlet:
	\[
	\pmb{X}\pmb{\beta} + \pmb{\e} = \pmb{y}
	\]
	A második Gauss-Markov feltétel szerint $\pmb{X}$ oszloprangja megegyezik oszlopainak számával, az oszlopok mind lineárisan függetlenek, azaz nincs zérus szinguláris értéke. Ezt $(\pmb{X}^T\pmb{X})^{-1}$ létezésénél már feltételeztük, formálisan ez is egyike a feltételeknek.
	\\
	\\
	A harmadik feltétel szerint
	\[
	\mathbb{E}[\pmb{\e} \mid \pmb{X}] = \pmb{0}
	\]
	\[
	\mathbb{E}
	\begin{bmatrix}
		\e_1 \mid \pmb{X} \\
		\e_2 \mid \pmb{X} \\
		\vdots \\
		\e_n \mid \pmb{X}
	\end{bmatrix}
	= \pmb{0}
	\]
	Ez azt jelenti, hogy a modell szerinti hibatag várható értékét nem befolyásolja egyik magyarázó változó sem. Ebből következőleg
	\[
	\mathbb{E}[\pmb{y} \mid \pmb{X}] = \mathbb{E}[\pmb{X}\pmb{\beta} + \pmb{\e} \mid \pmb{X}] = \pmb{X}\pmb{\beta}
	\]
	A negyedik feltétel a hibák kovariancia mátrixára vonatkozik, mégpedig
	\[
	\mathbb{E}[\pmb{\e}\pmb{\e}^T \mid \pmb{X}] = \sigma^2\pmb{I}
	\]
	A hibatagok \emph{homoszkedasztikusak és korrelálatlanok}, azaz azonosan $\sigma^2$ varianciájúak és $\forall i \ne j : Cov[\e_i, \e_j] = 0$. Ha kiírjuk $\pmb{\e}\pmb{\e}^T$ mátrixformáját:
	\[
	\mathbb{E}[\pmb{\e}\pmb{\e}^T \mid \pmb{X}] =
	\mathbb{E}
	\begin{bmatrix}
		\e_1^2 \mid \pmb{X} & \e_1 \e_2 \mid \pmb{X} & \dots & \e_1 \e_n \mid \pmb{X} \\
		\e_2^1 \mid \pmb{X} & \e_2 \e_2 \mid \pmb{X} & \dots & \e_2 \e_n \mid \pmb{X} \\
		\vdots & \vdots & \ddots & \vdots \\
		\e_n^1 \mid \pmb{X} & \e_n \e_2 \mid \pmb{X} & \dots & \e_n^2 \mid \pmb{X}
	\end{bmatrix}
	\]
	és persze $\forall i : \mathbb{E}[\e_i \mid \pmb{X}] = 0$ miatt a fenti mátrix diagonálisában $\e_i$-k varianciái, a többi helyen pedig a kovarianciák, amik a feltétel szerint $0$-k, így $\mathbb{E}[\pmb{\e}\pmb{e}^T \mid \pmb{X}]$ kovarianciamátrix valóban diagonális, a homoszkedaszticitás feltétele mellett pedig minden diagonális elem $\sigma^2$. Mostantól a hibatagok varianciáját $\pmb{\Sigma}$ fogja jelölni, $\pmb{\Sigma} = \sigma^2\pmb{I}$.
	\\
	\\
	Az utolsó feltétel szerint a hibatagok normális eloszlást követnek:
	\[
	\pmb{\e} \mid \pmb{X} \sim \mathcal{N}(\pmb{0}, \pmb{\Sigma})
	\]
	Kijelenthetjük tehát, hogy $y_i$-k varianciáját nem csak $\pmb{x}_i$-ek magyarázzák, hanem $\sigma^2$ \emph{magyarázatlan variancia} is. Úgy is megfogalmazhatjuk, hogy a modell szerint minden $\pmb{y}$ magyarázott változó-vektor regresszorok szerinti feltételes eloszlása
	\[
	\pmb{y} \mid \pmb{X} \sim \mathcal{N}(\pmb{X}\pmb{\beta}, \pmb{\Sigma})
	\]
	Lássuk be, hogy a feltételek teljesülése mellett $\ebeta$ valóban torzítatlan becslést ad $\pmb{\beta}$-ra! Láttuk, hogy $\ebeta = (\pmb{X}^T\pmb{X})^{-1}\pmb{X}^T\pmb{y}$, és a modell szerinti $\pmb{y} = \pmb{X}\pmb{\beta} + \pmb{\e}$ behelyettesítéssel
	\[
	\ebeta = (\pmb{X}^T\pmb{X})^{-1}\pmb{X}^T(\pmb{X}\pmb{\beta} + \pmb{\e})
	\]
	\[
	\ebeta = \pmb{\beta} + (\pmb{X}^T\pmb{X})^{-1}\pmb{X}^T\pmb{\e},
	\]
	mindkét oldalon véve a várható értéket:
	\[
	\mathbb{E}[\ebeta \mid \pmb{X}] = \mathbb{E}[\pmb{\beta} \mid \pmb{X}] + \mathbb{E}[(\pmb{X}^T\pmb{X})^{-1}\pmb{X}^T\pmb{\e} \mid \pmb{X}] = \pmb{\beta} + (\pmb{X}^T\pmb{X})^{-1}\mathbb{E}[\pmb{X}^T\pmb{\e}]
	\]
	Mivel a Gauss-Markov feltételekből következően $\mathbb{E}[\pmb{X}^T\pmb{\e}] = \pmb{0}$, így
	\[
	\mathbb{E}[\ebeta \mid \pmb{X}] = \pmb{\beta}
	\]
	ezzel készen is vagyunk. A $\mathbb{E}[\pmb{X}^T\pmb{\e}] = \pmb{0}$ tulajdonságot \emph{exogenitásnak} is hívjuk. Ez persze semmi mást nem jelent, mint hogy a regresszorok korrelálatlanok a hibával.
	\[
		Cov[\pmb{X}, \pmb{\e}] = \mathbb{E}[\pmb{X}^T\pmb{\e}] - \mathbb{E}[\pmb{X}]\mathbb{E}[\pmb{\e}] = \mathbb{E}[\pmb{X}^T\pmb{\e}] = \pmb{0}
	\]
	
	\subsection{$\ebeta$ varianciája}
	A hibavektor variancia-kovariancia mátrixához hasonlóan képezhetjük $\ebeta$ valószínűségi vektorváltozó variancia-kovariancia mátrixát:
	\[
	Var[\ebeta \mid \pmb{X}] = \mathbb{E}[(\ebeta - \pmb{\beta})(\ebeta - \pmb{\beta})^T \mid \pmb{X}]
	\]
	Láttuk, hogy
	\[
	\ebeta = \pmb{\beta} + (\pmb{X}^T\pmb{X})^{-1}\pmb{X}^T\pmb{\e} \Longrightarrow \ebeta - \pmb{\beta} = (\pmb{X}^T\pmb{X})^{-1}\pmb{X}^T\pmb{\e}
	\]
	\[
	\mathbb{E}[(\ebeta - \pmb{\beta})(\ebeta - \pmb{\beta})^T \mid \pmb{X}] = 
	\mathbb{E}\left[
	(\pmb{X}^T\pmb{X})^{-1}\pmb{X}^T\pmb{\e}((\pmb{X}^T\pmb{X})^{-1}\pmb{X}^T\pmb{\e})^T \mid \pmb{X}
	\right]
	\]
	A transzponálás "szorzatmegfordító" tulajdonságából következően, illetve $\pmb{X}^T\pmb{X}$ szimmetrikus voltából
	\[
	Var[\ebeta \mid \pmb{X}] = 
	\mathbb{E}\left[
	(\pmb{X}^T\pmb{X})^{-1}\pmb{X}^T\pmb{\e}\pmb{\e}^T\pmb{X}(\pmb{X}^T\pmb{X})^{-1} \mid \pmb{X}
	\right]
	\]
	\[
	Var[\ebeta \mid \pmb{X}] = (\pmb{X}^T\pmb{X})^{-1}\pmb{X}^T\mathbb{E}[\pmb{\e}\pmb{\e}^T \mid \pmb{X}]\pmb{X}(\pmb{X}^T\pmb{X})^{-1}
	\]
	Itt válik igazán fontossá, hogy $\mathbb{E}[\pmb{\e}\pmb{\e}^T \mid \pmb{X}]$ variancia-kovariancia mátrix alakja $\sigma^2\pmb{I}$, így $\sigma^2$ kiemelhető a mátrixszorzások elé, az identitást pedig triviálisan nem szükséges kiírni:
	\[
	Var[\ebeta \mid \pmb{X}] = \sigma^2(\pmb{X}^T\pmb{X})^{-1}\pmb{X}^T\pmb{X}(\pmb{X}^T\pmb{X})^{-1}
	\]
	A mátrixszorzás asszociativitásából pedig a
	\[
	Var[\ebeta \mid \pmb{X}] = \sigma^2(\pmb{X}^T\pmb{X})^{-1}
	\]
	végleges alakot kapjuk. Ugyanez megkapható az első fejezetben bemutatott $Var[\pmb{A}\pmb{\xi}] = \pmb{A}Var[\pmb{\xi}]\pmb{A}^T$ transzformált variancia képlettel is, $\pmb{\xi}$ helyett $\pmb{y}$, $\pmb{A}$ helyett pedig $(\pmb{X}^T\pmb{X})^{-1}\pmb{X}^T$ transzformáció mátrixxal (már ha $\pmb{X}$-eket fixnek tekintjük). A várható értékes felírásból látszik, hogy persze $Var[\ebeta \mid \pmb{X}]$ alakja
	\[
	\mathbb{E}[(\ebeta - \pmb{\beta})(\ebeta - \pmb{\beta})^T \mid \pmb{X}] =
	\begin{bmatrix}
		Var[\hat{\beta_1}] & Cov[\hat{\beta_1},\hat{\beta_2}] & \dots & Cov[\hat{\beta_1},\hat{\beta_p}] \\
		Cov[\hat{\beta_2},\hat{\beta_1}] & Var[\hat{\beta_2}] & \dots & Cov[\hat{\beta_2},\hat{\beta_p}] \\
		\vdots & \vdots & \ddots & \vdots \\
		Cov[\hat{\beta_p},\hat{\beta_1}] & Cov[\hat{\beta_p},\hat{\beta_2}] & \dots & Var[\hat{\beta_p}]
	\end{bmatrix}
	\]
	Ahogy $n \rightarrow \infty$, $\ebeta$ eloszlása \emph{aszimptotikusan normális lesz}, azaz
	\[
		\ebeta \mid \pmb{X} \sim \mathcal{AN}(\pmb{\beta}, \sigma^2(\pmb{X}^T\pmb{X})^{-1})
	\]
	Erről a következő alfejezetben részletesebben szó lesz.
	\\
	\\
	Csupán érdekesség, de el lehet képzelni, hogy heteroszkedaszticitás ($\exists i, j : \sigma_i^2 \ne \sigma_j^2$) és $p = 2$ mellett a modell az alábbi ábrával szemléltethető:
	\begin{center}
		\begin{tikzpicture}[
			declare function={
				normal(\x,\m,\s) = 1/(2*\s*sqrt(pi))*exp(-(\x-\m)^2/(2*\s^2));
			},
			declare function={invgauss(\a,\b) = sqrt(-2*ln(\a))*cos(deg(2*pi*\b));}
			]
			\begin{axis}[
				%no markers,
				domain=0:12,
				zmin=0, zmax=1,
				xmin=0, xmax=3,
				samples=200,
				samples y=0,
				view={40}{30},
				axis lines=middle,
				enlarge y limits=false,
				xtick={0.5,1.5,2.5},
				xmajorgrids,
				xticklabels={},
				ytick=\empty,
				xticklabels={$x_1$, $x_2$, $x_3$},
				ztick=\empty,
				xlabel=$x$, xlabel style={at={(rel axis cs:1,0,0)}, anchor=west},
				ylabel=$y$, ylabel style={at={(rel axis cs:0,1,0)}, anchor=south west},
				zlabel=$\epsilon$ sűrűségfüggvénye, zlabel style={at={(rel axis cs:0,0,0.5)}, rotate=90, anchor=south},
				set layers, mark=cube
				]
				
				\addplot3 [gray!50, only marks, mark=dot, mark layer=like plot, samples=200, domain=0.1:2.9, on layer=axis background] (x, {1.5*(x-0.5)+3+invgauss(rnd,rnd)*x}, 0);
				\addplot3 [samples=2, samples y=0, domain=0:3] (x, {1.5*(x-0.5)+3}, 0);
				\addplot3 [cyan!50!black, thick] (0.5, x, {normal(x, 3, 0.5)});
				\addplot3 [cyan!50!black, thick] (1.5, x, {normal(x, 4.5, 1)});
				\addplot3 [cyan!50!black, thick] (2.5, x, {normal(x, 6, 1.5)});
				
				\pgfplotsextra{
					\begin{pgfonlayer}{axis background}
						\draw [gray, on layer=axis background] (0.5, 3, 0) -- (0.5, 3, {normal(0,0,0.5)}) (0.5,0,0) -- (0.5,12,0)
						(1.5, 4.5, 0) -- (1.5, 4.5, {normal(0,0,1)}) (1.5,0,0) -- (1.5,12,0)
						(2.5, 6, 0) -- (2.5, 6, {normal(0,0,1.5)}) (2.5,0,0) -- (2.5,12,0);
						
					\end{pgfonlayer}
				}
			\end{axis}
		\end{tikzpicture}
	\end{center}

	\subsection{$\ebeta$ eloszlása}
	Láttuk, hogy
	\[
		\pmb{y} \mid \pmb{X} \sim \mathcal{N}(\pmb{X}\pmb{\beta}, \pmb{\Sigma})
	\]
	Mivel $\ebeta$ lineáris transzformációja $\pmb{y}$-nek, így a normális eloszlású valószínűségi változók transzformációs tulajdonságából adódóan és $\pmb{\Sigma} = \sigma^2\pmb{I}$-t kihasználva
	\[
		\ebeta \mid \pmb{X} \underset{n \rightarrow \infty}{\sim} \mathcal{N}((\pmb{X}^T\pmb{X})^{-1}\pmb{X}^T\pmb{X}\pmb{\beta}, \sigma^2(\pmb{X}^T\pmb{X})^{-1}\pmb{X}^T\pmb{X}(\pmb{X}^T\pmb{X})^{-1}) \sim \mathcal{N}(\pmb{\beta}, \sigma^2(\pmb{X}^T\pmb{X})^{-1})
	\] 
	azaz $\ebeta$ valóban normális eloszlást követ, a már jól ismert $\sigma^2(\pmb{X}^T\pmb{X})^{-1}$ varianciával és a valódi $\pmb{\beta}$ várható értékkel. Persze ez az $\ebeta$ \emph{vektorra} vonatkozott, és csak \emph{aszimptotikusan igaz}, azaz ha $n \rightarrow \infty$, hiszen ekkor a \emph{centrális határeloszlás tétel} értelmében a regresszorokból összetett mátrixkifejezés maga is normális lesz. Mi nyilván nem tudunk végtelen sok mintavétellel dolgozni, tehát azt mondjuk, hogy ha \emph{elég nagy $n$}, akkor a paraméterbecslés nagyon jól megközelíti a normális eloszlást.
	\\
	\\
	Gyakran $\ebeta$ aszimptotikus viselkedésére mint \emph{konzisztencia} utalnak, ez annyit jelent, hogy az eloszlás aszimptotikusan "ráhúzódik" a $\pmb{\beta}$ várható értékű normális eloszlásra. Formálisan felírva
	\[
		\ebeta \overset{p}{\rightarrow} \pmb{\beta} 
	\]
	\[
		\mathbb{P}(|\ebeta_i - \pmb{\beta}_i| < \varepsilon) > 1 - \delta, \quad \forall i, \quad \varepsilon, \delta > 0
	\]
	ahol $\overset{p}{\rightarrow}$ a \emph{probability limit}.
	Fontos, hogy a konzisztencia \emph{nem} vonja maga után a torzítatlanságot, hiszen gondoljuk meg, hogy ha $\frac{1}{n}$-et adunk egy torzítatlan konzisztens becsléshez, akkor a becslés továbbra is konzisztens marad, azonban már torzított lesz.
	\\
	\\
	$\ebeta$ \emph{feltétel nélküli varianciáját} az alábbi módon kaphatjuk meg:
	\[
		Var[\ebeta] = \mathbb{E}[Var[\ebeta \mid \pmb{X}]] + Var[\mathbb{E}[\ebeta \mid \pmb{X}]]
	\]
	ebből persze $\ebeta$ feltétel nélküli eloszlása is számolható lesz, de erre nem térünk ki.
	
	\subsection{Multikollinearitás}
	Ugyan feltettük, hogy nem létezik tökéletes multikollinearitás, de attól függetlenül valamilyen szintű multikollinearitás mindig elképzelhető a regresszorok között. Intuitíven a multikollinearitás egyfajta kapcsolatot vagy \emph{hasonlóságot, korrelációt} jelent a regresszorok között.
	\\
	\\
	Minél nagyobb a multikollinearitás mértéke, annál kevésbé különböznek $\pmb{X}$ oszlopai egymástól, azaz $\pmb{X}$ determinánsa annál kisebb. Emiatt $\pmb{X}^T\pmb{X}$ determinánsa is kisebb lesz, és mivel tetszőleges négyzetes mátrix esetén
	\[
		det(\pmb{A}^{-1}) = det(\pmb{A})^{-1},
	\]
	ezért a paraméterbecslés varianciája képletében $det((\pmb{X}^T\pmb{X})^{-1})$ nagy lesz. Ugyan ez nem egzakt matematikai összefüggés, de intuitíven el lehet képzelni, hogy ez "agresszívebb" $\ebeta$-varianciákat eredményez. Egy másik fontos következmény inkább technikai jellegű, mégpedig hogy a numerikus algoritmus, ami kiszámolja $\pmb{X}^T\pmb{X}$ inverzét, jelentős multikollinearitás mellett pontatlan eredményt fog adni. 
	\\
	\\
	Ugyan a multikollinearitás nem sérti meg a Gauss-Markov feltételeket, azaz még mindig $BLUE$ becslés lenne az $OLS$-el kapott $\ebeta$, nem is a legideálisabb a lehetségesen inflálódott paraméterbecslés-variancia és a numerikus számítások nehézsége miatt. Erre jelenthet megoldást az úgynevezett \emph{Ridge Regression} és \emph{Lasso Regression}, avagy rendre \emph{L2 és L1 regularizációs regresszió}, akit érdekel utánaolvashat, de erre nem térünk ki bővebben.
	
	\subsection{A hibavariancia becslése}
	A Gauss-Markov feltevések között szerepelt, hogy a hibatagok regresszorok szerinti feltételes eloszlása normális, egy bizonyos $\pmb{\Sigma}$ variancia-kovariancia mátrixxal. Azt is feltettük, hogy $\pmb{\Sigma}$ alakja
	\[
	\pmb{\Sigma} =
	\begin{bmatrix}
		\sigma^2 & 0 & \dots & 0 \\
		0 & \sigma^2 & \dots & 0 \\
		\vdots & \vdots & \ddots & \vdots \\
		0 & 0 & \dots & \sigma^2
	\end{bmatrix}
	\]
	azaz a mátrix diagonális, és minden diagonálisbeli elem azonosan $\sigma^2$. Felmerül persze a kérdés: Honnan tudjuk, hogy mi ez a $\sigma^2$ variancia? Ennek megoldásához \emph{torzítatlan becslést kell adnunk $\sigma^2$-ra a regresszióból}.
	\\
	\\
	$\sigma^2$ torzítatlan becslése:
	\[
	\widehat{\sigma^2} = \frac{\pmb{e}^T\pmb{e}}{n-p} = \frac{1}{n-p}\sum_{i=1}^{n}{(y_i - \hat{y}_i)^2}
	\]
	ahol $n$ a megfigyelések száma, $p$ pedig a magyarázó változók száma (az interceptet is beleértve). Mivel $p-1$ \emph{valódi magyarázó változónk} van (azaz ami nem konstans, azaz nem $\beta_0$), így a hibavariancia-becslés nevezőjében - a valódi ($\beta_1 \dots \beta_{p-1}$) $p-1$ darab magyarázó változóval - $n-(p-1)-1$ áll.
	
	\section{A $p=2$-es egyváltozós regresszió}
	Nézzük meg, hogy eddig látott paraméterbecslés és becslés-variancia hogy néz ki a legegyszerűbb, egy darab konstans interceptet és egy darab magyarázó változót tartalmazó OLS-el becsült modellben. A modell egyenlete minden $i = 1 \dots n$ megfigyelésre
	\[
	y_i = \beta_0 + \beta_1 x_i + \e_i
	\]
	Az $\pmb{X}$ design mátrixunk most
	\[
	\pmb{X}=
	\begin{bmatrix}
		1 & x_1 \\
		1 & x_2 \\
		\vdots & \vdots \\
		1 & x_n
	\end{bmatrix} \in \mathbb{R}^{n \times 2}
	\]
	lesz, $\ebeta$ paraméterbecslés pedig
	\[
	\ebeta = 
	\left(\begin{bmatrix}
		1 & 1 & \dots & 1 \\
		x_1 & x_2 & \dots & x_n
	\end{bmatrix}
	\begin{bmatrix}
		1 & x_1 \\
		1 & x_2 \\
		\vdots & \vdots \\
		1 & x_n
	\end{bmatrix}\right)^{-1}
	\begin{bmatrix}
		1 & 1 & \dots & 1 \\
		x_1 & x_2 & \dots & x_n
	\end{bmatrix}
	\begin{bmatrix}
		y_1 \\
		y_2 \\
		\vdots \\
		y_n
	\end{bmatrix}
	=
	\begin{bmatrix}
		n & \sum_i{x_i} \\
		\sum_i{x_i} & \sum_i{x_i^2}
	\end{bmatrix}^{-1}
	\begin{bmatrix}
		\sum_i{y_i} \\
		\sum_i{x_i y_i}
	\end{bmatrix}
	\]
	A $2 \times 2$-es mátrixok invertálása könnyen megy:
	\[
	\ebeta = 
	\frac{1}{n\sum_i{x_i^2} - (\sum_i{x_i})^2}
	\begin{bmatrix}
		\sum_i{x_i^2} & -\sum_i{x_i} \\
		-\sum_i{x_i} & n 
	\end{bmatrix}
	\begin{bmatrix}
		\sum_i{y_i} \\
		\sum_i{x_i y_i}
	\end{bmatrix}
	=
	\frac{1}{n\sum_i{x_i^2} - (\sum_i{x_i})^2}
	\begin{bmatrix}
		\sum_i{x_i^2}\sum_i{y_i} - \sum_i{x_i}\sum_i{x_i y_i} \\
		-\sum_i{x_i}\sum_i{y_i} + n\sum_i{x_i y_i}
	\end{bmatrix}
	=
	\]
	\[
	=
	\begin{bmatrix}
		\frac{n(\frac{1}{n}\sum_i{x_i^2}) \cdot n(\frac{1}{n}\sum_i{y_i})-n(\frac{1}{n}\sum_i{x_i}) \cdot n(\frac{1}{n}\sum_i{x_i y_i})}{n^2(\frac{1}{n}\sum_i{x_i^2})-n^2(\frac{1}{n}\sum_i{x_i})^2} \\
		\frac{n^2\frac{1}{n}\sum_i{x_i y_i} - n(\frac{1}{n}\sum_i{x_i}) \cdot n(\frac{1}{n}\sum_i{y_i})}{n^2(\frac{1}{n}\sum_i{x_i^2})-n^2(\frac{1}{n}\sum_i{x_i})^2}
	\end{bmatrix}
	\]
	Az $n$ elemű mintából képzett \emph{mintaátlag} semmi más, mint $\frac{1}{n}\sum_i{x_i}$ illetve $\frac{1}{n}\sum_i{y_i}$, a kovariancia $x$ és $y$ között pedig $\mathbb{E}[xy] - \mathbb{E}[x]\mathbb{E}[y]$, $n$ elemű - a várható értéket torzítatlanul becsülő - mintaátlagokkal ez persze semmi más, mint az \emph{empirikus kovariancia} $ empcov[x,y] = \frac{1}{n}\sum_i{x_i y_i} - (\frac{1}{n}\sum_i{x_i})(\frac{1}{n}\sum_i{y_i})$. $x$ varianciája $\mathbb{E}[x^2] - \mathbb{E}^2[x]$-ként áll elő, $\mathbb{E}[x^2]$ empirikus becslése pedig $\frac{1}{n}\sum_i{x_i^2}$. A vektor mindkét elemében $n^2$-el leosztva látható, hogy a nevezőkben pontosan $x$ mintából számolt varianciája ($empvar$) van, míg a vektor második elemének számlálója pontosan $x$ és $y$ mintából számolt kovarianciája. A vektor első elemének számlálójában $\bar{x^2} \cdot \bar{y} - \bar{x} \cdot \bar{xy}$ áll. Jelölje mostantól a mintából számolt varianciát és kovarianciát $\widehat{Var}$ és $\widehat{Cov}$, ezzel a paraméterbecslés alakja
	\[
	\ebeta =
	\begin{bmatrix}
		\frac{\bar{x^2} \cdot \bar{y} - \bar{x} \cdot \bar{xy}}{\widehat{Var}[x]} \\
		\frac{\widehat{Cov}[x,y]}{\widehat{Var}[x]}
	\end{bmatrix} 
	\]
	Azt kaptuk tehát, hogy a legegyszerűbb egyváltozós regresszió becsült paraméterei
	\[
	\hat{\beta_0} = \frac{\bar{x^2} \cdot \bar{y} - \bar{x} \cdot \bar{xy}}{\widehat{Var}[x]}	
	\]
	\[
	\hat{\beta_1} = \frac{\widehat{Cov}[x,y]}{\widehat{Var}[x]}
	\]
	Sokszor a mintaszámmal normálatlan empirikus kovarianciát és varianciát \emph{$S_{xy}$} és \emph{$S_{xx}$} jelöléssel látják el:
	\[
	S_{xy} = \sum_{i=1}^{n}{(x_i - \bar{x})(y_i - \bar{y})}
	\]
	\[
	S_{xx} = \sum_{i=1}^{n}{(x_i - \bar{x})^2},
	\]
	ezekkel felírva $\beta_1$ becslését:
	\[
	\hat{\beta_1} = \frac{S_{xy}}{S_{xx}}
	\]
	$\beta_0$ becslésének alakja $\beta_1$ ismeretében is kiszámolható, és sokszor ez a módszer sokkal kényelmesebb (már ha ismerjük $\hat{\beta_1}$ értékét):
	\[
	\hat{\beta_0} = \bar{y} - \hat{\beta_1}\bar{x}
	\]
	Ez nem csak intuitívan értelmezhető ("Az átlagos $y$ semmi más, mint az y-tengellyel való metszéspont és $\hat{\beta_1}\bar{x}$ összege"), hanem formálisan is levezethető a modell egyenletéből (meg abból, hogy beláttuk, hogy a paraméterbecslés torzítatlan a feltevéseink mellett, illetve hogy a hibatagok várható értéke $0$):
	\[
	y = \beta_0 + \beta_1 x + \e
	\]
	\[
	\mathbb{E}[y] = \beta_0 + \beta_1\mathbb{E}[x]
	\]
	\[
	\beta_0 = \mathbb{E}[y] - \beta_1 \mathbb{E}[x]
	\]
	A várhatóérték-operátor helyett persze a mintaátlagokkal dolgozva:
	\[
	\beta_0 = \bar{y} - \beta_1 \bar{x}
	\] 
	valóban.
	\subsection{$\ebeta$ varianciája egyváltozós regresszió esetén}
	Láttuk, hogy a paraméterbecslés varianciája az általános esetben
	\[
	Var[\ebeta] = \sigma^2(\pmb{X}^T\pmb{X})^{-1}
	\]
	A már levezetett $p=2$-es design mátrixxal dolgozva:
	\[
	Var[\ebeta] = \sigma^2
	\left(\begin{bmatrix}
		1 & 1 & \dots & 1 \\
		x_1 & x_2 & \dots & x_n
	\end{bmatrix}
	\begin{bmatrix}
		1 & x_1 \\
		1 & x_2 \\
		\vdots & \vdots \\
		1 & x_n
	\end{bmatrix}\right)^{-1}
	=
	\sigma^2
	\frac{1}{n\sum_i{x_i^2} - (\sum_i{x_i})^2}
	\begin{bmatrix}
		\sum_i{x_i^2} & -\sum_i{x_i} \\
		-\sum_i{x_i} & n 
	\end{bmatrix}
	\]
	Használjuk ki az empirikus variancia képletét:
	\[
	n\sum_{i=1}^{n}{x_i^2} - (\sum_{i=1}^{n}{x_i})^2 = n\sum_{i=1}^{n}{(x_i - \bar{x})^2}
	\]
	Innen könnyen látszik, hogy
	\[
	Var[\hat{\beta_0}] = \sigma^2\frac{\sum_{i=1}^{n}{x_i^2}}{n\sum_{i=1}^{n}{(x_i - \bar{x})^2}}
	\]
	\[
	Var[\hat{\beta_1}] = \sigma^2\frac{1}{\sum_{i=1}^{n}{(x_i - \bar{x})^2}}
	\]
	Kimondhatjuk tehát, hogy ahogy $\sigma^2$ nő, úgy nő a paraméterbecslésünk varianciája, avagy \emph{bizonytalansága} is. Hasonlítsuk össze az általános esetben kapott $\ebeta$ variancia képletét $\beta_1$ varianciáéval:
	\[
	\sigma^2(\pmb{X}^T\pmb{X})^{-1}
	\]
	\[
	\sigma^2(\sum_{i=1}^{n}{(x_i - \bar{x})^2})^{-1}
	\]
	A $2 \times 2$-es mátrixszorzást elvégezve tényleg azt kaptuk, hogy az egyváltozós regresszió esetén $S_{xx}$ semmi más, mint az $\pmb{X}^T\pmb{X}$ centralizálatlan regresszor-kovariancia mátrix.
	\\
	\\
	Nagyon fontos - és ezért itt is kihangsúlyozandó - hogy \emph{y varianciája kettő forrásból jön: a regresszorok varianciájából és a regresszorok által nem magyarázott hibavarianciából}. Írjuk ezt az összefüggést fel a mi esetünkben a modellegyenlet segítségével (persze a regresszorok és a hibák korrelálatlansága mellett):
	\[
	Var[\pmb{y}] = \beta_1^2Var[\pmb{x}] + Var[\pmb{\e}] 
	\]
	Itt kihasználtuk, hogy a modell szerint $\beta_0$ konstans, így zérus varianciája van. $Var[\pmb{\e}]$ hibavariancia az a része $y$ varianciájának, amit nem magyaráznak a regresszorok. Ha $Var[\pmb{\e}]$ kicsi, ez annyit jelent, hogy a becsült $\hat{y}$-ok és a tényleges $y$-ok közel vannak egymáshoz, azaz a regresszióval nagyon jól becsülhetjük a valódi $y$ értékeket.
	
	\section{Az $R^2$ mutató}
	Tekintsük az egyváltozós regressziós modellt. Legyen
	\[
	R^2 := \frac{\beta_1^2Var[\pmb{x}]}{Var[\pmb{y}]}
	\]
	az arány, amiben a regresszorok varianciája magyarázza a magyarázott változó teljes varianciáját. $R^2$ $0$ és $1$ közötti szám, minél közelebb van $1$-hez, annál jobban becsülhető $y$ a regresszorokkal. $\beta_1$ becslését beírva adódik:
	\[
	R^2 = \frac{|Cov[\pmb{x},\pmb{y}]|^2}{Var[\pmb{x}]Var[\pmb{y}]}
	\]
	$R^2$ a regresszió "erősségét" mutatja, így a normálatlan empirikus kovarianciákkal és varianciákkal ($S_{xy}$, $S_{xx}$, $S_{yy}$):
	\[
	R^2 = \frac{S^2_{xy}}{S_{xx}S_{yy}}
	\]
	Itt persze $S_{yy} = \sum_i{(y_i - \bar{y})^2}$
	Vezessük be az alábbi jelöléseket:
	\[
	SST = \sum_{i=1}^{n}{(y_i - \bar{y})^2}
	\]
	\[
	SSE = \sum_{i=1}^{n}{(\hat{y}_i - \bar{y})^2}
	\]
	\[
	SSR = \sum_{i=1}^{n}{e_i^2}
	\]
	$SST$ a \emph{Sum of Squares Total}, $SSE$ a \emph{Sum of Squares Explained}, $SSR$ pedig a \emph{Sum of Squares Residual}. Az előbbi varianciafelbontásból könnyen látszik, hogy mivel $SSE$ a regresszorok által magyarázott variancia, $SSR$ pedig a magyarázatlan variancia:
	\[
	SST = SSE + SSR
	\]
	$R^2$-et az előbbihez hasonlóan, csak most az új jelölésekkel felírva:
	\[
	R^2 = \frac{SSE}{SST} = 1- \frac{SSR}{SST}
	\]
	(Az irodalomban néha - zavaró módon - Az $SSE$ a hibák négyzetösszegét jelenti, mint Sum of Squares Error, és az $SSR$ jelenti a magyarázott varianciát, mint Sum of Squares Regression.)
	\section{Kihagyott változó bias - Omitted Variable Bias}
	Tekintsünk egy 
	\[
		y = \beta_0 + \beta_1 x + \beta_2 z + \e
	\]
	lineáris modellt. Ahhoz, hogy létezzen kihagyott változó bias, a kihagyott változó együtthatója nem lehet zérus, illetve a kihagyott változónak \emph{korrelálnia kell} egy másik, regresszióban szereplő magyarázó változóval.
	\\
	\\
	Tegyük fel, hogy kihagyjuk $z$-t a regresszióból:
	\[
		y = \tilde{\beta_0} + \tilde{\beta_1} x + \tilde{\e}
	\]
	és hogy $z$-t $x$ a következőképpen magyarázza:
	\[
		z = \delta_0 + \delta_1 x + \nu
	\]
	Helyettesítsük be a második egyenletet az eredeti teljes egyenletbe:
	\[
		y = (\beta_0 + \beta_2 \delta_0) + (\beta_1 + \beta_2 \delta_1)x + (\e + \beta_2 \nu)
	\]
	Látható, hogy ha ezen a kihagyott változós modellen végeznénk el a regressziós paraméterbecslést, $x$ együtthatójának nem $\beta_1$-et, hanem $\beta_1 + \beta_2 \delta_1$-et kapnánk, ami nyilvánvalóan az eredeti modellel konzisztensen \emph{torzított}. Úgy is gondolhatunk erre, hogy a kihagyott $z$ miatt $x$ becsült együtthatója tartalmazni fogja az indirekt hatást is ($z$-n $x$ hatása $\delta_1$, ezt megszorozva még $y$-n $z$ hatásával).
	\\
	\\
	Mátrixformában az Omitted Variable Bias az alábbi formában szemléltethető. Legyenek
	\[
		\pmb{X} =
		\begin{bmatrix}
		x_1 \\
		x_2 \\
		\vdots \\
		x_n
		\end{bmatrix}, \quad
		\pmb{Z} =
		\begin{bmatrix}
		z_1 \\
		z_2 \\
		\vdots \\
		z_n 
		\end{bmatrix}
	\]
	a regresszorokat tartalmazó vektorok. A $z$-t kihagyó modell design mátrixa pusztán $\pmb{X}$, így az ebből nyert paraméterbecslés
	\[
		\ebeta = (\pmb{X}^T\pmb{X})^{-1}\pmb{X}^T\pmb{y}
	\] 
	Írjuk be $\pmb{y}$ helyére a tényleges, teljes modellből származó alakot:
	\[
		\ebeta = (\pmb{X}^T\pmb{X})^{-1}\pmb{X}^T(\pmb{X}\pmb{\beta} + \pmb{Z}\pmb{\delta} + \pmb{\e}) = \pmb{\beta} + (\pmb{X}^T\pmb{X})^{-1}\pmb{X}^T\pmb{Z}\pmb{\delta} + (\pmb{X}^T\pmb{X})^{-1}\pmb{X}^T\pmb{\e}
	\]
	Mindkét oldalon várható értéket véve, és visszaemlékezve arra, hogy az utolsó tag zérus lesz:
	\[
		\mathbb{E}[\ebeta] = \pmb{\beta} + (\pmb{X}^T\pmb{X})^{-1}\mathbb{E}[\pmb{X}^T\pmb{Z}]\pmb{\delta}
	\]
	ahol látható, hogy a jobb oldal második tagja pontosan a kihagyott $z$ változó miatti torzítás, avagy \emph{bias}.
	
	\section{MSE és a bias-variancia tradeoff}
	Tekintsünk egy általános
	\[
		\pmb{y} = \textswab{P}(\pmb{X}) + \pmb{\e}
	\]
	modellt. Csakúgy, mint eddig, $\pmb{X}$ a regresszorok, $\pmb{y}$ a magyarázott változó vektora, $\textswab{P}$ pedig valamilyen függvény. Az $\pmb{\e}$ hibák regresszorok szerinti feltételes várható értéke $0$. Figyeljük meg, hogy a lineáris regresszió esetében $\textswab{P}$ a lineáris $\pmb{\beta}$ együtthatóvektor. A célunk, hogy megtaláljuk azt a $\widehat{\textswab{P}}$ függvényt, amivel a becsült
	\[
		\hat{\pmb{y}} = \widehat{\textswab{P}}(\pmb{X})
	\]
	$\hat{\pmb{y}}$-ok és a tényleges $\pmb{y}$-ok négyzetes távolsága a lehető legkisebb. Nyilvánvalóan - ezt a regressziónál is láttuk már - egy olyan $\widehat{\textswab{P}}$-t találni, ami \emph{tökéletesen} becsüli $\pmb{y}$-t reális esetben lehetetlen, így fontos lesz, hogy valahogyan számszerűsíthessük a megfigyeléseken (mintán) alapuló illetve a még megfigyeletlen regresszorokon vett várható tévedésünket.
	\\
	\\
	A \emph{Mean Squared Error}, röviden \emph{MSE} klasszikusan az átlagos avagy várható négyzetes eltérések összegét jelenti a prediktált $\hat{\pmb{y}}$ és a tényleges $\pmb{y}$-ok között. Attól függően azonban, hogy mit akarunk vele pontosan kifejezni, definiálhatjuk a \emph{prediktorok} (a fenti "klasszikus" eltérés-négyzetösszeges definíció) és a \emph{becslések} szemszögéből is. 
	\\
	\\
	A \emph{prediktorok} szemszögéből a definíció egy $n$ elemű mintán
	\[
		MSE := \frac{1}{n}\sum_{i=1}^{n}{(y_i - \hat{y}_i)^2}
	\]
	Kompaktabban kifejezve a tényleges $\pmb{e}$ hibákkal:
	\[
		MSE := \frac{1}{n}\pmb{e}^T\pmb{e}
	\]
	Ha $\textswab{P}$ becsléséhez nem használtuk fel az összes $n$ elemet, hanem csak $m < n$-et, akkor az $MSE$ definiálható úgy is, mint az átlagos négyzetes hiba a becsléshez fel nem használt adatpontokon:
	\[
		MSE := \frac{1}{n-m}\sum_{i=m+1}^{n}{(y_i - \hat{y}_i)^2}
	\]
	\\
	\\
	A \emph{becslés} szemszögéből az $MSE$ a $\widehat{\textswab{P}}$ becslésünkre vonatkozik, mégpedig egy teoretikus valódi $\textswab{P}$ függvény mellett
	\[
		MSE(\widehat{\textswab{P}}) = \mathbb{E}_{\textswab{P}}[(\widehat{\textswab{P}} - \textswab{P})^2]
	\]
	Ez semmi más, mint a \emph{második momentuma} a $\widehat{\textswab{P}} - \textswab{P}$ becslés-eltérésnek. Ebből a definícióból következik a \emph{bias-variancia tradeoff}, melynek fontos következményei lesznek. Lássuk ezt be!
	\\
	\\
	Tudjuk, hogy tetszőleges $\xi$ valószínűségi változóra $\mathbb{E}[\xi^2] = Var[\xi] + \mathbb{E}^2[\xi]$. Most $\xi = \widehat{\textswab{P}} - \textswab{P}$-vel:
	\[
		MSE = \mathbb{E}[(\widehat{\textswab{P}} - \textswab{P})^2] = Var[\widehat{\textswab{P}} - \textswab{P}] + \mathbb{E}^2[\widehat{\textswab{P}} - \textswab{P}]
	\]
	A $\mathbb{E}[\widehat{\textswab{P}} - \textswab{P}]$ várható eltérést ($\textswab{P}$-hez képest) hívjuk \emph{bias}-nak, avagy \emph{torzításnak}, ennek négyzetére $Bias^2[\widehat{\textswab{P}}]$-ként hivatkozunk mostantól. Mivel $\textswab{P}$ a modell szerint egy konkrét függvény, így $Var[\widehat{\textswab{P}} - \textswab{P}] = Var[\widehat{\textswab{P}}]$, és ezzel
	\[
		MSE = \mathbb{E}[(\widehat{\textswab{P}} - \textswab{P})^2] = Var[\widehat{\textswab{P}}] + Bias^2[\widehat{\textswab{P}}].
	\]
	A becslés szemszögéből tehát az $MSE$ semmi más, mint a becslés varianciájának és torzítás-négyzetének összege. Ezt az összefüggést hívjuk bias-variancia tradeoffnak, hiszen adott $MSE$ mellett ha az egyiket csökkenteni is tudom, a másik nőni fog. Komplex $\widehat{\textswab{P}}$ becslés mellett a bias, avagy torzítottság alacsony lesz, azonban magas varianciája, avagy \emph{bizonytalansága} lesz a becslésemnek. Egyszerű $\widehat{\textswab{P}}$ mellett pedig a bias lesz magas, alacsony varianciával.
	
	\section{A heteroszkedaszticitás kezelése, a GLS eljárás}
	A Gauss-Markov feltevések egyike volt, hogy $\pmb{\Sigma}$ hiba variancia-kovariancia mátrix diagonális, és a diagonális elemek azonosan $\sigma^2$-ek. Láttuk azt is, hogy ezekre a $\sigma^2$-ekre torzítatlan becslést ad a $\widehat{\sigma^2}$ \emph{hibavariancia becslés}. Azt az esetet, amikor $\pmb{\Sigma}$ diagonális, azonban $\sigma^2$-ek nem egyenlőek, heteroszkedaszticitásnak hívjuk, és emellett a hiba variancia-kovariancia mátrix mellett a paraméterbecslés varianciája már nem a megszokott
	\[
		Var[\ebeta] = \sigma^2(\pmb{X}^T\pmb{X})^{-1}
	\]
	alakú, hiszen nem emelhettük ki $\sigma^2\pmb{I}$-t középről.
	\\
	\\
	Tudjuk, hogy minden variancia-kovariancia mátrix szimmetrikus és pozitív szemidefinit. Ezért $\exists \pmb{P} : \pmb{P}\pmb{P}^T = \pmb{\Sigma}$ (ez a \emph{Cholesky-felbontás}), tehát felbonthatjuk a kovariancia mátrixot kettő, $\pmb{\Sigma}$-val azonos dimenziójú invertálható mátrix szorzatára (Ez analóg azzal, hogy $\mathbb{R}$-en minden pozitív szemidefinit (nemnegatív) valós számnak létezik négyzetgyöke, és a négyzetgyök csak akkor $0$, ha maga a szám $0$, de most a zérus kovariancia mátrix esetétől eltekintünk).
	\\
	\\
	A célunk az, hogy $\pmb{\Sigma}$ kovariancia mátrixot $\sigma^2\pmb{I}$ alakúra hozzuk. Ha megszorozzuk balról $\pmb{P}^{-1}$-el a $\pmb{\e}$ hibatagot, a hiba varianciája:
	\[
		Var[\pmb{P}^{-1}\pmb{\e}] = \pmb{P}^{-1}\pmb{\Sigma}{\pmb{P}^{-1}}^T
	\]
	A felbontásból következően, és a ${\pmb{P}^T}^{-1} = {\pmb{P}^{-1}}^T$ összefüggést felhasználva:
	\[
		\pmb{P}^{-1}\pmb{\Sigma} = \pmb{P}^T \Longrightarrow \pmb{P}^{-1}\pmb{\Sigma}{\pmb{P}^{-1}}^T = \pmb{P}^T{\pmb{P}^{-1}}^T = \pmb{I}
	\]
	Azt kaptuk tehát, hogy ha a $\pmb{P}^{-1}$-el beszorzott módosított regressziós modellegyenletet tekintjük
	\[
		\pmb{P}^{-1}\pmb{X}\pmb{\beta} + \pmb{P}^{-1}\pmb{\e} = \pmb{P}^{-1}\pmb{y}
	\]
	akkor ebben a modellben a hiba varianciamátrixa az identitás mátrix, így nem áll fenn heteroszkedaszticitás.
	\\
	\\
	A módosított modellel való paraméterbecslés tehát
	\[
		\ebeta = ((\pmb{P}^{-1}\pmb{X})^T(\pmb{P}^{-1}\pmb{X}))^{-1}(\pmb{P}^{-1}\pmb{X})^T\pmb{P}^{-1}\pmb{y}
	\]
	\[
		\ebeta = (\pmb{X}^T{\pmb{P}^{-1}}^T\pmb{P}^{-1}\pmb{X})^{-1}\pmb{X}^T{\pmb{P}^{-1}}^T\pmb{P}^{-1}\pmb{y}
	\]
	Szintén a felbontásból, most már mátrixhatványokkal kiírva adódik, hogy
	\[
		\pmb{P} = \pmb{\Sigma}^{\frac{1}{2}}
	\]
	Így
	\[
		{\pmb{P}^{-1}}^T\pmb{P}^{-1} = {\pmb{\Sigma}^{-\frac{1}{2}}}^T\pmb{\Sigma}^{-\frac{1}{2}} = \pmb{\Sigma}^{-1}
	\]
	Ezzel a paraméterbecslés alakja:
	\[
		\ebeta_{GLS} = (\pmb{X}^T\pmb{\Sigma}^{-1}\pmb{X})^{-1}\pmb{X}^T\pmb{\Sigma}^{-1}\pmb{y}
	\]
	Ez már konzisztens a Gauss-Markov feltételekkel, így $\ebeta$ paraméterbecslés teljesíti a BLUE kritériumokat. Ez az eljárás egy speciális esete a \emph{Generalized Least Squares (GLS)} becslési eljárásnak, ahol $\pmb{\Sigma}$ nemdiagonális elemei mind $0$-k, az angol irodalomban \emph{Weighted Least Squares} néven szerepel. $\pmb{\Sigma}^{-1}$-t, azaz az inverz variancia-kovariancia mátrixot \emph{precíziós mátrixnak} is hívják. A $GLS$ működik autokorreláció esetén is, azaz tetszőleges pozitív definit $\pmb{\Sigma}$ hibavariancia mátrixxal is, és igazából ez az amit "hivatalosan" $GLS$-nek hívnak.
	
	\subsection{*A GLS becslés analitikus levezetése}
	$\ebeta_{GLS}$ alakját megkaphatjuk úgy is, ha tekintjük az alábbi optimalizációs problémát:
	\[
		(\pmb{y} - \pmb{X}\pmb{b})^T\pmb{\Sigma}^{-1}(\pmb{y} - \pmb{X}\pmb{b}) \rightarrow \underset{\pmb{b}}{argmin}
	\]
	Ez persze semmi más, mint a Mahalanobis távolság minimalizálása $\pmb{y}$ és $\pmb{X}\pmb{b}$ között $\pmb{b}$ szerint. Kibontva a kifejezést és a $\pmb{b}$ szerinti deriváltat $0$-ra állítva ($\pmb{b} = \pmb{\ebeta_{GLS}}$):
	\[
		2\pmb{X}^T\pmb{\Sigma}^{-1}\pmb{X}\ebeta_{GLS} - 2\pmb{X}^T\pmb{\Sigma}^{-1}\pmb{y} = 0
	\]
	Ebből
	\[
		\ebeta_{GLS} = (\pmb{X}^T\pmb{\Sigma}^{-1}\pmb{X})^{-1}\pmb{X}^T\pmb{\Sigma}^{-1}\pmb{y}
	\]
	Így is megkaptuk ugyanazt az alakot.
	\subsection{A Feasible Generalized Least Squares (FGLS) eljárás}
	Ha nem ismerjük a valódi $\pmb{\Sigma}$ hibavariancia-kovariancia mátrixot, akkor a már bevezetett $\widehat{\sigma^2}$ becsült varianciákkal konstruálhatjuk meg a becsült $\widehat{\pmb{\Sigma}}$ mátrixot.
	\\
	\\
	Az $FGLS$ eljárás \emph{kétlépcsős}, első lépésként először is elvégzünk a módosítatlan $\pmb{X}\pmb{\beta} + \pmb{\e} = \pmb{y}$ modellel egy egyszerű $OLS$ becslést, melyből $\ebeta_{OLS}$-t kapjuk (ez persze heteroszkedaszticitás esetén nem BLUE becslés). Az így kapott
	\[
		\pmb{e} = \pmb{y} - \pmb{X}\ebeta_{OLS}
	\]
	hibavektorokkal megbecsüljük $\widehat{\pmb{\Sigma}}$ hibavariancia-kovariancia mátrixot. Persze mivel heteroszkedaszticitást feltételeztünk, így $\widehat{\sigma_1^2}, \dotsm \widehat{\sigma_n^2}$ becsült hibavarianciákat csupán egyelemes mintával (rendre $e_1, \dots, e_n$ tényleges hibákkal) becsülhetnénk, azaz a tényleges becsült varianciákhoz \emph{valamilyen előzetes feltevés a heteroszkedaszticitással konzisztens hibavarianciákra}, de ezzel részletesebben nem foglalkozunk, és feltesszük, hogy "valahogy" meg tudjuk kapni ezen becsléseket. Így a variancia-kovariancia mátrix:  

	\[
		\widehat{\pmb{\Sigma}} = 
		\begin{bmatrix}
			\widehat{\sigma_1^2} & 0 & \dots & 0 \\
			0 & \widehat{\sigma_2^2} & \dots & 0 \\
			\vdots & \vdots & \ddots & \vdots \\
			0 & 0 & \dots & \widehat{\sigma_n^2}
		\end{bmatrix}
	\]
	Második lépésként az első lépésben kapott $\widehat{\pmb{\Sigma}}$ mátrixxal $GLS$ becsléssel megkapjuk a
	\[
		\ebeta_{FGLS} = (\pmb{X}^T\widehat{\pmb{\Sigma}}^{-1}\pmb{X})^{-1}\pmb{X}^T\widehat{\pmb{\Sigma}}^{-1}\pmb{y}
	\]
	FGLS paraméterbecslést. Ez az eljárás \emph{iterálható}, azaz vehetjük az $FGLS$ becslésből kapott
	\[
		\pmb{e}_{FGLS} = \pmb{y} - \pmb{X}\ebeta_{FGLS}
	\]
	hibavektort, és újrabecsülhetjük $\widehat{\pmb{\Sigma}}$-t:
	\[
		\widehat{\pmb{\Sigma}}_{FGLS} =
		\begin{bmatrix}
			\widehat{\sigma_{FGLS,1}^2} & 0 & \dots & 0 \\
			0 & \widehat{\sigma_{FGLS,2}^2} & \dots & 0 \\
			\vdots & \vdots & \ddots & \vdots \\
			0 & 0 & \dots & \widehat{\sigma_{FGLS,n}^2}
		\end{bmatrix}
	\]
	Ezzel az újrabecsült kovariancia-variancia mátrixxal az új paraméterbecslésünk
	\[
		\ebeta_{FGLS2} = (\pmb{X}^T\widehat{\pmb{\Sigma}}_{FGLS}^{-1}\pmb{X})^{-1}\pmb{X}^T\widehat{\pmb{\Sigma}}_{FGLS}^{-1}\pmb{y}
	\]
	Az iteráció tetszőlegesen sokáig folytatódhat, és minden iterációval egyre közelebb kerülünk a tényleges $\pmb{\beta}$-hoz.
	
	\chapter{Paraméterszignifikancia-tesztek lineáris regresszió esetén}
	A lineáris regresszió tanulmányozása folyamán fontos kitérni a paraméterbecslések \emph{szignifikanciájára}, azaz arra a kérdésre, hogy jelentősen csökken-e a $\ebeta$-val való predikciós/magyarázó erő, ha $\ebeta$ egy vagy több elemét $0$-nak vesszük. A szignifikancia tesztelése minden esetben \emph{hipotézisvizsgálat}, a különbség a nullhipotézisek megfogalmazása között van, és az így különböző velejáró tesztstatisztika-eloszlásokban. 
	
	\section{t-teszt egyelemes paraméterrestrikcióra}
	Láttuk, hogy $\ebeta$ elég közel lesz a normális eloszláshoz megfelelően sok megfigyelés mellett (mostantól mindig fix $\pmb{X}$-el dolgozunk). Azt is láttuk, hogy a hibavariancia torzítatlan becslése $\widehat{\sigma^2} = \frac{\pmb{e}^T\pmb{e}}{n-p}$ lesz (ugyan ezt nem bizonyítottuk de akit érdekel utánanézhet ha nagyon unatkozik).
	\[
		\ebeta \sim \mathcal{N}(\pmb{\beta}, \widehat{\sigma^2}(\pmb{X}^T\pmb{X})^{-1})
	\]
	Jelölje $(\pmb{X}^T\pmb{X})^{-1}$ inverz centralizálatlan (és normálatlan) regresszor-kovariancia mátrix $k$-adik diagonális elemét $\pmb{L}_{k}^2$, ez persze semmi más, mint a $0 \le k \le p$-adik paraméterbecslés varianciájának és a becsült magyarázatlan $\widehat{\sigma^2}$ hibavarianciának hányadosa. A null- és alternatív hipotéziseink:
	\begin{center}
		\begin{tabular}{|c|c|}
			\hline
			$H_0$ & $H_1$ \\
			\hline
			$\beta_k = 0$ & $\beta_k \ne 0$ \\
			\hline
		\end{tabular}
	\end{center}
	azaz hogy a nullhipotézis alatt a k-adik paraméterbecslésünk igazi értéke értéke $0$. A tesztstatisztikánk:
	\[
		t = \frac{\ebeta_k - 0}{\widehat{\sigma}\pmb{L}_{k}} = \frac{\ebeta_k}{\pmb{L}_k\sqrt{\frac{1}{n-p}\pmb{e}^T\pmb{e}}}
	\]
	Mivel $\pmb{y}$-ok normális eloszlásúak az $\pmb{X}\pmb{\beta}$ várható értékeik körül, ezért $\pmb{e}^T\pmb{e}$ normális eloszlású valószínűségi változó négyzetösszege, azaz $\chi^2$ eloszlású.
	\\
	\\
	Még mielőtt továbbmennénk, lássuk be, hogy $\ebeta$ független $\pmb{e}^T\pmb{e}$-től.
	\[
		\ebeta = (\pmb{X}^T\pmb{X})^{-1}\pmb{X}^T\pmb{y}
	\]
	Ha $\pmb{e}$ tényleges hibát az $\pmb{X}\pmb{X}^{\dagger}$ oszloptér-vetítés mátrixxal írjuk föl, és meggondoljuk, hogy persze $\pmb{I} - \pmb{X}\pmb{X}^{\dagger}$ szintén vetítés mátrix, csak az $\pmb{X}$ oszlopterére ortogonális vektortérre:
	\[
		\pmb{e} = (\pmb{I} - \pmb{X}\pmb{X}^{\dagger})\pmb{y} = \pmb{y} - \pmb{X}\ebeta
	\] 
	ebből mátrixszorzásokkal és $\pmb{y}^T\pmb{y} = \sigma^2$-el megkaphatjuk $\ebeta^T\pmb{e}$-t:
	\[
		\ebeta^T\pmb{e} = \sigma^2(\pmb{X}^{\dagger} - (\pmb{X}^T\pmb{X})^{-1}\pmb{X}^T\pmb{X}\pmb{X}^{\dagger}) = \sigma^2\pmb{0} = 0
	\]
	Tehát a paraméterbecslésünk valóban független a hibáktól, ezért a tesztstatisztikánkban a $\chi^2$ és a $\mathcal{N}$ eloszlások függetlenek, azaz
	\[
		t = \frac{\ebeta_k - 0}{\widehat{\sigma}\pmb{L}_{k}} = \frac{\ebeta_k}{\pmb{L}_k\sqrt{\frac{1}{n-p}\pmb{e}^T\pmb{e}}} \sim t_{n-p}
	\]
	t-eloszlást követ, $n-p$ szabadságfokkal. A hipotézisvizsgálat szokásos módszertana szerint definiálunk egy $\alpha$ szignifikanciaszintet, és megnézzük, hogy $t$ beleesik-e az $\alpha$ által meghatározott elfogadási tartományba. Ha beleseik, nem tudjuk elutasítani $H_0$-t, azaz $\ebeta_k$-ról \emph{nem mondhatjuk, hogy nem $0$}. Ha kívül esik, akkor $\ebeta_k$ \emph{szignifikáns (szignifikánsan eltér $0$-tól) egy $\alpha$ szignifikancia szint mellett}.
	
	\section{F-teszt többszörös paraméterrestrikcióra}
	Ha egyszerre több paraméter \emph{közös szignifikanciáját} szeretnénk vizsgálni (például az első $m+1$ paraméterét), akkor a hipotéziseink:
	\begin{center}
		\begin{tabular}{|c|c|}
			\hline
			$H_0$ & $H_1$ \\
			\hline
			$\beta_0 = \beta_1 = \dots = \beta_m = 0$ & $\beta_i \ne 0\quad \forall i = 0 \dots,m$ \\
			\hline
		\end{tabular}
	\end{center}
	Legyen $u$ az \emph{unrestricted}, avagy \emph{teljes} modellünk, ahol egyik paraméterünk sem $0$. Legyen $r$ a \emph{restricted}, avagy \emph{korlátozott} modellünk, ahol most speciálisan az első $m+1$ paraméterünk $0$ (ez a nullhipotézis melletti modell). Jelölje $SSR_u$ és $SSR_r$ rendre az ezen modellek melletti Sum of Squared Residualsokat, avagy hibanégyzetösszegeket. A tesztstatisztikánk
	\[
		F = \frac{\frac{SSR_r - SSR_u}{m+1}}{\frac{SSR_u}{n-p}}
	\]
	ahol $p$ a teljes modell magyarázó paramétereinek száma, $m+1 < p$ pedig a teljes és korlátozott modellek paraméterszámának különbsége. Gondoljuk meg, hogy a tesztstatisztika sosem lehet negatív, hiszen a teljes modell mellett mindig kisebb lesz a hibanégyzetösszeg (több paraméterrel biztosan jobban fogjuk tudni magyarázni a magyarázott változók varianciáját, kérdés persze, hogy \emph{nem magyarázzuk-e túl} azt).
	$SSR_r - SSR_u$ és $SSR_u$ $\chi^2$ eloszlásúak, rendre $n-(p-(m+1))-(n-p) = n-p+m+1-n+p = m+1$ és $n-p$ szabadságfokokkal, tehát a tesztstatisztikánk \emph{F-eloszlást követ}:
	
	\[
		F = \frac{\frac{SSR_r - SSR_u}{m+1}}{\frac{SSR_u}{n-p}} \sim F_{m+1,n-p}
	\]
	A t-teszttel analóg módon itt is megkeressük az ilyen paraméterezésű F-eloszlásból $\alpha$ szignifikancia szint mellett a kritikus értékeket, így az elfogadási tartományt is megtaláljuk, és ha $F$ beleesik ebbe, akkor nem tudjuk elvetni a nullhipotézist, azaz \emph{a korlátozott modell nem magyarázza $\pmb{y}$ varianciáját szignifikánsan rosszabban, mint a teljes modell}, így elhagyható a modellből az első $m+1$ magyarázó változó. Fontos kiemelni, hogy ebből csakis az első $m+1$ paraméter \emph{közös szignifikanciájára} következtethetünk, egyenként semmit nem tudunk meg róluk.
	\subsection{Az F-teszt és a t-teszt ekvivalenciája}
	Ha az F-tesztben pontosan egy $\beta$-t veszünk $0$-nak a nullhipotézis alatt, ez megegyezik az adott $\beta$-ra vonatkozó t-teszttel, mégpedig a tesztstatisztikákkal felírva:
	\[
		F_{1,n-p} = t^2_{n-p}
	\]
	\chapter{A Maximum Likelihood Estimation (MLE)}
	A lineáris regresszióbeli paraméterbecslésünk egy másik, elterjedt módja az úgynevezett $MLE$, avagy \emph{Maximum Likelihood Estimation} eljárás. A "likelihood" a megfigyelt adatok valószínűsége valamilyen paraméterek függvényében, az $MLE$ ezt a valószínűséget mint egy szélsőértékfeladatot maximalizálva találja meg azokat a paramétereket, amikkel ez a valószínűség a megfigyelt adatokon a lehető legnagyobb.
	\\
	\\
	A lineáris regresszió modellünk továbbra is
	\[
		\pmb{y} = \pmb{X}\pmb{\beta} + \pmb{\e}
	\]
	Éljünk továbbra is azzal a feltétellel, hogy $\pmb{\e}$ normális eloszlású $0$ várható értékkel és $\sigma^2$ varianciával. Az $OLS$-ről szóló fejezetben megnéztük, hogy a modellről alkotott feltételeink mellett $\pmb{y}$ valószínűségi vektorváltozónk eloszlása
	\[
		\pmb{y} \mid \pmb{X} \sim \mathcal{N}(\pmb{X}\pmb{\beta}, \pmb{\Sigma})
	\]
	ahol $\pmb{\Sigma}$ diagonális, és minden diagonálisbeli elem azonosan $\sigma^2$, azaz nincs heteroszkedaszticitás. Írjuk fel annak a valószínűségét, hogy valamilyen $\ebeta$ és $\sigma^2$ paraméterek mellett az $i$-edik megfigyelt magyarázott változót kaptuk az $i$-edik megfigyelt magyarázóvektorból a modell mellett:
	\[
		\mathbb{P}(y_i \mid \pmb{x}_i, \ebeta, \sigma^2) = \frac{1}{\sigma\sqrt{2\pi}}exp \left(
			-\frac{1}{2\sigma^2}(y_i - \ebeta^T\pmb{x}_i)^2
		\right)
	\] 
	Mivel minden megfigyelésünk független, annak a valószínűsége tehát, hogy mind az $n$ darab $y_i$-t kaptuk:
	\[
		\mathbb{P}(\{y_i\}_{i=1}^{n} \mid \{\pmb{x}_i\}_{i=1}^{n}, \ebeta, \sigma^2) = \prod_{i=1}^{n}{
		\frac{1}{\sigma\sqrt{2\pi}}exp \left(
		-\frac{1}{2\sigma^2}(y_i - \ebeta^T\pmb{x}_i)^2
		\right)	
		}
	\]
	Valószínűségi vektorváltozókkal kompaktabban kiírva ugyanezt:
	\[
		\mathbb{P}(\pmb{y} \mid \pmb{X}, \ebeta, \sigma^2) = |\pmb{\Sigma}|^{-1/2}(2\pi)^{-n/2}exp\left(
			-\frac{1}{2}(\pmb{y} - \pmb{X}\ebeta)^T\pmb{\Sigma}^{-1}(\pmb{y} - \pmb{X}\ebeta)
		\right)
	\]
	Mivel $\pmb{\Sigma} = \sigma^2\pmb{I}$, így a fenti formula az alábbival egyenértékű:
	\[
		\mathbb{P}(\pmb{y} \mid \pmb{X}, \ebeta, \sigma^2) = (2\sigma^2\pi)^{-n/2}exp\left(
			-\frac{1}{2\sigma^2}(\pmb{y} - \pmb{X}\ebeta)^T(\pmb{y} - \pmb{X}\ebeta)
		\right)
	\]
	Ez a parametrizált feltételes valószínűség az $L(\ebeta, \sigma^2)$ \emph{likelihood függvény}, melynek logaritmusát véve $l(\ebeta, \sigma^2)$ \emph{log-likelihood függvényt} kapjuk.
	\[
		log(\mathbb{P}(\pmb{y} \mid \pmb{X}, \ebeta, \sigma^2)) =: l(\ebeta, \sigma^2)
	\] 
	Mostantól az argumentumok beírásának elhagyásával egyszerűen $L$ és $l$-ként fogunk hivatkozni erre.
	
	\section{A $\ebeta_{ML}$ és $\sigma^2_{ML}$ paraméterbecslések}
	Az $MLE$ célja, hogy ezt az $L$ likelihood függvényt \emph{maximalizálja} $\ebeta$-ban és $\sigma^2$-ben. Ha feltesszük, hogy ismerjük $\sigma^2$-et, akkor természetesen a maximalizáció csak $\ebeta$-ban lesz. Mivel szorzatot nehéz differenciálni, és mivel a $log$ függvény szigorúan monoton nő, így az $L$ függvény maximalizálása helyett $l$-et fogjuk maximalizálni. $l$ alakja
	\[
		l = -\frac{n}{2}log(2\sigma^2\pi) - \frac{1}{2\sigma^2}(\pmb{y} - \pmb{X}\ebeta)^T(\pmb{y} - \pmb{X}\ebeta)
	\]
	Tegyük fel, hogy ismerjük $\sigma^2$-et. Formálisan felírva $\ebeta$-t megkaphatjuk az alábbi módon (az összeg első első tagja most konstans):
	\[
		\ebeta_{ML} = \underset{\ebeta}{argmax}\left(
			-\frac{1}{2\sigma^2}(\pmb{y} - \pmb{X}\ebeta)^T(\pmb{y} - \pmb{X}\ebeta)
		\right)
	\]
	Mivel a konstans szorzó nem befolyásolja a szélsőérték feladat optimális megoldását, így a mínusz előjelet elhagyva és minimalizációs problémára átírva:
	\[
		\ebeta_{ML} = \underset{\ebeta}{argmin}\left(
			(\pmb{y} - \pmb{X}\ebeta)^T(\pmb{y} - \pmb{X}\ebeta)
		\right)
	\]
	Ha összehasonlítjuk ezt az $OLS$ becslés szélsőérték feladatával, látjuk, hogy \emph{pontosan ugyanazt kaptuk} (csak ott $\pmb{e}^T\pmb{e}$-el volt felírva). Kimondhatjuk tehát, hogy \emph{ha normális eloszlású hibát feltételezünk a lineáris regressziós modellben, és ezen kívül minden egyéb feltétel szintén teljesül, akkor az $MLE$ becslés eredményének ugyanazt kapjuk, mint az $OLS$ becslés eredménye, $\ebeta_{ML} = \ebeta_{OLS}$}.
	\\
	\\
	A Maximum Likelihood becslés azonban nem csak $\ebeta$ becslését tudja megadni, hanem $\sigma^2$-ét is! Ha most megfordítjuk a helyzetet, és feltesszük, hogy már ismerjük $\ebeta_{ML}$-t, akkor a fenti szélsőértékfeladat már $\sigma^2$-ről szól:
	\[
		\sigma^2_{ML} = \underset{\sigma}{argmax}\left(
		-\frac{n}{2}log(2\sigma^2\pi) - \frac{1}{2\sigma^2}(\pmb{y} - \pmb{X}\ebeta_{ML})^T(\pmb{y} - \pmb{X}\ebeta_{ML})
		\right)
	\]
	Ezt megoldva kapjuk $\sigma^2$ $MLE$ becslését:
	\[
		\widehat{\sigma^2}_{ML} = \frac{1}{n}(\pmb{y} - \pmb{X}\ebeta_{ML})^T(\pmb{y} - \pmb{X}\ebeta_{ML})
	\]
	Ez persze semmi más, mint
	\[
		\widehat{\sigma^2}_{ML} = \frac{1}{n}\pmb{e}^T\pmb{e}
	\]
	ahol $\pmb{e}$-k a $\ebeta_{ML}$-el becsült $\hat{y}_i$-k és a tényleges $y_i$-k hibavektora. Felmerülhet a kérdés, hogy miért $n$-el, és nem $n-1$-el osztjuk le $\pmb{e}^T\pmb{e}$-t a varianciabecslésnél, mint eddig. Az $n$-el való leosztás valóban nem eredményez torzítatlan varianciabecslést, azonban emlékezzünk vissza a \emph{Mean Squared Error (MSE)} definíciójára, most $\widehat{\sigma^2}_{ML}$-re nézve:
	\[
		MSE = \frac{\sum_{i=1}^{n}{(\widehat{\sigma^2}_{ML} - \sigma^2)^2}}{n}
	\]
	Ha $n-1$-el osztottunk volna le, ugyan torzítatlan becslésünk lenne, de \emph{az $MSE$ nagyobb lenne}. Mivel $MSE = Bias^2 + Var$, így ugyan csökkentettük $Bias^2$-et (így persze $Bias$-t is), de ezzel \emph{növeltük} a becslés varianciáját. Az, hogy melyik a jobb vagy rosszabb, nagyon függ attól a becslési feladat kontextusától, és általánosan nem is könnyű az, hogy "melyik rossz a kevésbé rossz".  
	\\
	\\
	A hibanormalitási és az egyéb feltételek mellett tehát az $MLE$ becslés az $OLS$-el együtt $BLUE$. Fontos, hogy a Maximum Likelihood Estimation technika nem csak regressziós modellek esetén alkalmazható, ezért is különösen fontos nekünk a hibanormalitás. 
	\\
	\\
	Ha nem tesszük fel $\pmb{y}$ normális eloszlását, akkor az $MLE$ becslés  \emph{hatásosabb} az OLS-nél, és numerikus kiszámítása is stabilabb (nem kell magas dimenziójú mátrixokat invertálgatnunk - multikollinearitás esetén láttuk, hogy az $OLS$ becslés kiszámítása ott problémákba tud ütközni. Összességében tehát a feltételeink mellett (a hibanormalitást és $\pmb{y}$ normális eloszlását is hozzávéve persze) kijelenthető, hogy a kettő megegyezik.
	
\end{document}
